% ************************** Thesis Abstract *****************************

\begin{abstract}

%In particular the maturation of single-cell RNA-sequencing technologies has enabled the identification of transcriptional profiles associated in a myriad of biological processses, including lineage diversification and cell fate commitment.
%Consequently, the profiling of the epigenome at the single-cell resolution level is receiving increasing attention, but without associated transcriptomic readouts, the conclusions that can be extracted from epigenetic measurements are limited.\\

Single-cell profiling techniques have provided an unprecented opportunity to study cellular heterogeneity at multiple molecular levels. This represents a remarkable advance over traditional bulk sequencing methods, particularly to study lineage diversification and cell fate commitment events in heterogeneous biological processes, including the immune system, embryonic development and cancer.\\ 
The maturation of single-cell RNA-sequencing technologies has enabled the identification of transcriptional profiles, but the accompanying epigenetic changes and the role of other molecular layers in driving cell fate decisions remains poorly understood.\\
More recently, technological advances enabled multiple biological layers to be probed in parallel at single-cell resolution, unveling a powerful approach for investigating regulatory relationships. Such single-cell multi-modal technologies can reveal multiple dimensions of cellular heterogeneity and uncover how this variation is coupled between the different molecular layers, hence enabling a more profound mechanistic insight than can be inferred by analysing a single data modality.\\
The increasing availability of multi-modal data sets needs to be accompanied by the development of novel integrative strategies. Yet, the high levels of missing information, the inherent amounts of technical noise and the potentially large number of cells, makes the integrative analysis of single-cell omics one of the most challenging problems in computational biology. \\

In this PhD thesis we propose an experimental methodology and a computational framework for the integrative study of multiple omics in single cells.\\

In Chapter 1 we present single-cell nucleosome, methylation and transcription sequencing (scNMT-seq), an experimental protocol for the genome-wide profiling of RNA expression, DNA methylation and chromatin accessibility in single cells. In a proof of concept application, we validate the quality of the readouts and we compare it against similar technologies. Finally, we show how scNMT-seq can be used to study coordinated epigenetic and transcriptomic heterogeneity along a simple differentiation process, hence expanding our ability to investigate the dynamics of the epigenome across cell fate transitions.\\

In Chapter 2 we discuss Multi-Omics Factor Analysis (MOFA), a statistical framework for the integration of multi-omics data sets. MOFA is a latent variable model that offers a principled approach to interrogate multi-omics data sets in a completely unsupervised manner, revealing the underlying sources of sample heterogeneity. Once the model is trained, the infered low-dimensional space can be queried using a toolkit of downstream analysis, including visualisation, clusteirng, imputation or prediction of clinical outcomes.\\
First, we validate the different features of the model using simulated data. Second, we demonstrate the potential of MOFA in a study of 200 chronic lymphocytic leukaemia patients. In a quick unsupervised analysis, MOFA revealed the most important dimensions of disease heterogeneity and connected them to clinical markers that are commonly used in practice. In a second application we show how MOFA can reveal biological insights from noisy single-cell multi-modal data.\\

In Chapter 3 we use scNMT-seq to generate the first triple-omics roadmap of mouse gastrulation. Using MOFA, we perform an integrative study of all molecular layers, revealing novel insights on the dynamics of the epigenome. Notably, we show that cells commited to mesoderm and endoderm undergo widespread epigenetic rearrangements, driven by demethylation in enhancer marks and by concerted changes in chromatin accessibility. In contrast, the epigenetic landscape of ectoderm cells remains in a “default” state, resembling earlier stage epiblast cells is epigenetically established in the early epiblast. This work provides a comprehensive insight into the molecular logic for a hierarchical emergence of the primary germ layers, hence revealing underlying molecular constituents of the Waddington's landscape.\\

In Chapter 4 we propose an improved formulation of the MOFA framework with the aim of performing integrative analysis of large-scale single-cell data sets across multiple studies/conditions as well as data modalities.\\
To tailor MOFA to the statistical challenges of single-cell data, we introduce key methodological developments, including a fast stochastic variational inference framework, a new structured sparsity prior and the relaxation of the assumption of independent samples.\\
First, we benchmark the new features of the model using simulated data. Next, we use a single-cell DNA methylation data set of neurons from mouse frontal cortex to demonstrate how from a seemingly unimodal data set, one can investigate hypothesis using a multi-group and multi-view setting. Finally, we apply MOFA to the scNMT-seq data set generated in Chapter3, disentangling the sources of heterogeneity assocaited with early cell fate decisions.\\

Finally, Chapter 5 summarises this thesis and provides an outlook of future research.

\end{abstract}
