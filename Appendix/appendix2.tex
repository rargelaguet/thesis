\graphicspath{{Appendix/appendix_figures/}}

\chapter{Characterisation of MOFA factors in the scNMT-seq gastrulation data set} \label{appendix:mofa_gastrulation}

% TO REWRITE
\begin{figure}[H]
	\centering
	\includegraphics[width=0.95\linewidth]{mofa_gastrulation_factor3}
	\caption[]{\textbf{Characterisation of MOFA Factor 3 as antero-posterior axial patterning and mesoderm maturation.} \\
	(a) Beeswarm plot of Factor 3 values, grouped and coloured by cell type. The mesoderm cells are subclassified into nascent and mature mesoderm. \\
	(b) RNA expression weights for Factor 3. Genes with large positive weights increase expression in the positive factor values (more anterior), whereas genes with negative weights increase expression in the negative factor values (more posterior).\\
	(c) Same beeswarm plots as in (a), coloured by the relative RNA expression of genes with the highest positive (top) or negative (bottom) weight.\\
	(d) Gene set enrichment analysis of the gene weights of Factor 3. Shown are the top most significant pathways from MSigDB C2 \cite{Subramanian2005,Ashburner2000}.
	}
	\label{fig:mofa_gastrulation_factor3}
\end{figure}


% TO REWRITE
\begin{figure}[H]
	\centering
	\includegraphics[width=0.95\linewidth]{mofa_gastrulation_factor4}
	\caption[]{\textbf{Characterisation of MOFA Factor 4 as notochord formation.} \\
		(a) Beeswarm plot of Factor 4 values, grouped and coloured by cell type. The endoderm cells are subclassified into notochord (dark green) and not notochord (green) (see Figure S2). \\
		(b) RNA expression weights for Factor 4. Genes with large positive weights increase expression in the positive factor values (endoderm cells), whereas genes with negative weights increase expression in the negative factor values (notochord cells).\\
		(c) Same beeswarm plots as in (a), coloured by the relative RNA expression of genes with the highest negative weight (notochord markers).
	}
	\label{fig:mofa_gastrulation_factor4}
\end{figure}


% TO REWRITE
\begin{figure}[H]
	\centering
	\includegraphics[width=0.95\linewidth]{mofa_gastrulation_factor5}
	\caption[]{\textbf{Characterisation of MOFA Factor 5 as mesoderm patterning.} \\
	(a) Beeswarm plot of Factor 5 values, grouped and coloured by cell type. \\
	(b) RNA expression weights for Factor 5. A higher absolute value indicates higher feature importance.\\
	(c) Same beeswarm plots as in (a), coloured by the relative RNA expression of genes with the highest weight on this factor.}
	\label{fig:mofa_gastrulation_factor5}
\end{figure}


% TO REWRITE
\begin{figure}[H]
	\centering
	\includegraphics[width=0.95\linewidth]{mofa_gastrulation_factor6}
	\caption[]{\textbf{Characterisation of MOFA Factor 6 as cell cycle.} \\
	(a) Beeswarm plot of Factor 6 values, grouped by cell type and coloured by infered cell cycle state using \textit{cyclone}\cite{Scialdone2015}.\\
	(b) RNA expression weights for Factor 6. Genes with large positive weights increase expression in the positive factor values (G1/S phase), whereas genes with negative weights increase expression in the negative factor values (G2/M phase).}
	\label{fig:mofa_gastrulation_factor6}
\end{figure}
