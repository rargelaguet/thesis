\graphicspath{{Chapter3/Figs/}}

\chapter{Multi-omics profiling of mouse gastrulation at single-cell resolution}

In this chapter I will describe a study where we combined scNMT-seq (Chapter 1) and MOFA (Chapter 2) to explore the relationship between the transcriptome and the epigenome during mouse gastrulation.

The work discussed in this chapter results from a collaboration with the group of Wolf Reik (Babraham Institute, Cambridge, UK). It has been peer-reviewed and published in \cite{Argelaguet2019}. The experiments were carried out by Stephen Clark, Hisham Mohammed and Carine Stapel, with the help of Wendy Dean and Courtney Hanna for the collection of embryos. Tim Lohoff prepared the Embryoid Body \textit{TET} TKO culture. Wei Xie and Yunlong Xiang shared the ChIP-seq data that was used to define germ layer-specific enhancers. Felix Krueger processed and managed sequencing data. Christel Krueger processed the ChIP-seq data. I performed the majority of the computational analysis, but with contributions from some authors. In particular, Stephen Clark calculated the transcription factor motif enrichment analysis, Carine Stapel explored the neuroectoderm and pluripotency signatures in ectoderm enhancers, and Ivan Imaz-Rosshandler performed the mapping to the gastrulation atlas. John C. Marioni, Oliver Stegle and Wolf Reik supervised the project. The article was jointly written by Stephen Clark, Carine Stapel and me, with input from all authors.

\section{Introduction}

The human body is composed of a myriad of cell types with specialised structure, organisation and function; and yet, each cell in the body contains the same genetic information. The modulation of the genetic code by internal and external factors begin during embryonic development, giving rise to the formation of specialised molecular patterns that ultimately determines the complexity of adult organisms \cite{Rosalind2018}. A key phase in mammalian embryonic development is gastrulation, when a single-layered blastula of pluripotent and relatively homogeneous cells is reorganised to form the three primordial germ layers: the ectoderm, mesoderm and endoderm \cite{Tam1997, Solnica-Krezel2012, Tam2007}.

The onset of gastrulation is determined by the formation of the primitive streak, which establishes the initial bilateral symmetry of the body. Involution of epiblast cells through the primitive streak gives rise to the mesoderm and endoderm, whereas epiblast cells establish the ectoderm \cite{Arnold2009,Tam2007,Tam1997,Tam1993}. Although differences exist between species, the morphogenic process of gastrulation is evolutionary conserved throughout the animal kingdom \cite{Solnica-Krezel2012}. In most cases, gastrulation is characterised by an epithelial to mesenchymal transition that brings mesodermal end endoderm progenitors beneath the future ectoderm. The epiblast cells that did not migrate through the primitive streak differentiate towards ectoderm, which eventually gives rise to the nervous system (neural ectoderm) and epidermis (surface ectoderm). The embryonic endoderm gives rise to the interior linings of the digestive tract, the respiratory tract, the urinary bladder and part of the auditory system. The embryonic mesoderm gives rise to muscles, connective tissues, bone, cartilage, blood, kidneys, among others. 
% The cell fate map of embryonic cells has been an area of intensive research \cite{XX}. Early work in model organisms such as \textit{Drosophila melanogaster} identified morphogenic gradients that partition the embryo into multiple compartments, and specific combinations of extracellular molecules will induce differential signalling cascades that will commit cells to the different cell type fates.
% More recently, advances in CRIPSR-Cas9 genome editing have revolutionised lineage tracing methods. By introducing genetic scars with high-information content, this technology promises a big leap towards the ultimative goal of creating a high-resolution fate map for every cell in the embryo \cite{McKenna2019,Chan2019}.

\subsection{Transcriptomic studies}

Significant research effort has been deployed to understand the molecular changes underlying gastrulation. Historically, microscopy was used to quantify gene expression at single cell resolution. However, constraints imposed by fluorophore emission spectra made this approach unsuitable for genome-wide studies. Only after the breakthrough made by the introduction of single-cell sequencing technologies it has been possible to generate comprehensive molecular roadmaps of embryonic development \cite{Schaum2018,PijuanSala2019,Cao2019,Aviv2017}. In a pioneer study, \cite{PijuanSala2019} generated the first high-resolution atlas of gastrulation and early somitogenesis by profiling the RNA expression of 116,312 cells from 411 whole mouse embryos collected between E6.5 and E8.5. This effort completed earlier attempts of reconstructing the transcriptomic landscape of post-implantation embryos \cite{Mohammed2017,Scialdone2016,Ibarra-Soria2018,Wen2017}. At the same time, another study employed a more scalable methodology  to profile around 2 million cells from 61 embryos ranging from E9.5 and 13.5 days of gestation, spanning early organogenesis \cite{Cao2019}. By constructing a densely sampled reference data set, both works have laid the ground for understanding transcriptomic variation during development. 


\subsection{Epigenetic studies}

RNA expression is a big and central piece in the puzzle of understanding embryonic development, but still a single piece. The next step is to connect this information to the accompanying epigenetic changes, which are becoming more accessible to profile with single-cell technologies. In differentiated cell types, epigenetic marks confer stable characteristic patterns of cell type identity which have been extensively profiled using bulk sequencing approaches. Nevertheless, because of the low amounts of input material and the extensive cellular heterogeneity, the study of the epigenetic landscape during early development remains poorly understood \cite{Kelsey2017}.

\subsubsection{Pre-implantation: establishment of the pluripotent state}

The first efforts to interrogate the epigenetic dynamics using (bulk) next generation sequencing technologies have provided valuable insights for the pre-implantation stage. Multiple studies have described that, after fertilisation, there is a round of reprogramming that resets the epigenetic landscape to a ground state \cite{Smith2012,Lee2014}. DNA methylation is globally removed and the chromatin attains its highest levels of accessibility \cite{Wu2016}. Consistently, Hi-C experiments have suggested a flexible chromatin landscape, with lack of topologically associating domains (TADs) or chromatin compartments \cite{Ke2017,Du2017,Tee2014}, providing a plausible explanation for the remarkably plasticity of pluripotent ESCs.\\
In contrast to DNA methylation, the presence of post-translational modifications in histone marks are abundant at this stage, potentially providing the major mechanism of epigenetic regulation \cite{Hanna2018,Tee2014}. Several histone modifications have been studied in ESCs, the most prominent being H3K27ac and H3K4me3, both (generally) activatory marks; and H3K27me3 and H3K9me3, both (generally) repressive marks \cite{Zhao2015}. Interestingly, many genes that are silenced in ESCs contain both activatory (H3K4me3) and repressive (H3K27me3) epigenetic marks. This distinctive signature of ESCs is thought to establish a bivalent or poised signature for a transcriptionally-ready state for genes that become expressed after gastrulation \cite{Bernstein2006,Tee2014}. 

\subsubsection{Post-implantation: exit of pluripotency}

In post-implantation development, cells exit pluripotency and undergo a set of critical cell fate decisions that will ultimately give rise to all somatic cell types. While multiple studies have profiled the epigenetic landscape in pre-implantation embryos, the epigenetic landscape of gastrulation and early mammalian organogenesis remains largely unexplored.

DNA methylation is one of the few epigenetic marks that has been profiled in a genome-wide manner, both at the bulk level and at the single cell level \cite{Auclair2014,Zhang2017,Dai2016,Rulands2018}. All studies found that the hypomethylated state in E3.5 blastocysts is followed by a \textit{de novo} DNA methylation wave upon implantation (between E4.5 and E5.5) that leads to a hypermethylation of most of the genome. The increase in DNA methylation is concomitant with the increased deposition of repressive histone marks, presumably with the aim of restricting the differentiation potential of early pluripotent cells \cite{Atlasi2017}.\\
The \textit{de novo} methyltransferases (DNMT3A and DNMT3B) are the enzymes responsible for the insertion of DNA methylation marks. Both genes are highly expressed in early mouse embryos, and catalytically inactive mutants of both enzymes lacked \textit{de novo} methylation activity \cite{Auclair2014,Okano1999}. Interestingly, mouse ESCs remain viable despite complete loss of DNA methylation, but they are uncapable of undergoing cell fate commitment and remain in the pluripotent state \cite{Tsumura2006}.

The interplay of histone marks during post-implantation development is complex and remains poorly understood. H3K4me3 is detected at transcription start sites after the zygotic genome activation, and remains remarkably stable across different pluripotency stages as well as in differentiated cell types \cite{Heintzman2009}. H3K4me3 is thought to facilitate transcription by inducing a more efficient assembly of the transcriptional machinery \cite{Atlasi2017,Vastenhouw2010}. The other conventional activatory mark, H3K27ac, is deposted in different types of regulatory elements, including promoters and enhancers. It is significantly more dynamic than H3K4me3 in response to internal and external stimuli, and is hence a stronger candidate to regulate cell fate transitions \cite{Atlasi2017,Rada-Iglesias2011}.\\
The inhibitory mark H3K27me3 shows a marked increase upon implantation, deposited by the Polycomb repressive complex 2 (PRC2) around multiple regulatory elements, including CpG-rich promoters of developmental genes. H3K27me3 is often present in transcriptionally inactive regions with low levels of DNA methylation, suggesting a potential antagonism between H3K27me and DNA methylation \cite{Brinkman2012,Atlasi2017}. Interestingly, inactivating PRC2 components in mouse embryos does not affect pre-implantation development, but the embryos become unviable after gastrulation\cite{Shan2017}. This suggests that H3K27me3 has a critical role in regulating gene expression during cell fate commitment after germ layer specification.

\subsubsection{Gastrulation: germ layer specification}

The post-implantation blastocyst is a relatively homogeneous population of cells and can be characterised to some accuracy by bulk sequencing approaches. However, germ layer specification is uniquely heterogeneous and extremely challenging to study without single-cell technologies. Despite the technical difficulties, some studies have been reported where the authors attempted to manually dissect each germ layer, followed by bulk sequencing \cite{Zhang2018}. This revealed that the relatively homogeneous epigenetic landscape at the epiblast is succedeed by a more dynamic landscape, driven by the emergence of regulatory elements that become activated in a lineage-specific manner \cite{Zhang2018,Lee2015}. Consistent with a role of DNA methylation during gastrulation, perturbations that target the Ten-eleven translocation (TET) family of dioxygenases display developmental defects related to germ layer specification, ranging from impaired migration of primitive streak cells to failed maturation of the mesoderm layer \cite{Dai2016}.

The recent development of single-cell multi-modal technologies, where epigenomes can be unequivocally assigned to transcriptomes at single-cell resolution, unveils novel opportunities to study the cell fate commitment events during gastrulation. These methodologies have been successful in pre-implantation stages \cite{Guo2017,Wang2019,Liu2019}, but gastrulation has remained elusive.

