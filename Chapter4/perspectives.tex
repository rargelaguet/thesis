\section{Limitations and open perspectives}

In this Chapter we proposed a generalisation of the MOFA model for the principled analysis of large-scale structured data sets. Although we have emphasised single-cell applications, the model remains applicable to bulk data sets.\\
MOFA+ solves some of the limitations of the MOFA model presented in Chapter 2, but a significant amount of challenges remain unsolved and could be addressed in future research:

\begin{itemize}
	\item \textbf{Linearity}: this is arguably the major limitation of MOFA. Although it is critical for obtaining interpretable feature weights, this results in a significant loss of explanatory power. Deep generative models have proven successful in modelling complex observations. Their principle is the use of non-linear maps via neural networks to encode the parameters of probability distributions. Among this class of methods, variational autoencoders provide a rigorous and scalable non-linear generalisation of factor models. \cite{Ainsworth2018}.

	\item \textbf{Improving the stochastic inference scheme}: a common extension of stochastic gradient descent is the addition of a \textit{momentum} term, which has been widely adopted in the training of artificial neural networks \cite{Zeiler2012,Ning1999}. The idea is to take account of past updates when calculating the present step, using for example a moving average calculation. This has been shown to improve the stability of gradients vectors, thus leading to a faster convergence.

	\item \textbf{Modelling dependencies between groups}: often groups are not independent and have some type of structure among themselves. A clear example are time course experiments. Explicit modelling of these dependencies, when known, could help on model inference and interpretation.

	\item \textbf{Modelling continuous dependencies between samples and/or features}: in the MOFA framework the views and the groups correspond to discrete and non-overlapping sets. An interesting improvement would be to model continuous dependencies using Gaussian Process priors \cite{Casale2018}. A clear application for this is spatial transcriptomics, where one could build a covariance matrix using spatial distances which can then be imposed in the prior distribution of the latent factors (recall that in MOFA and MOFA+ the prior distribution for the factors assumes independence between samples). This would improve the detection of sources of variation with a spatial component.
\end{itemize}