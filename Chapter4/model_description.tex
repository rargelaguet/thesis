\graphicspath{{Chapter4/Figs/simulations/}{Chapter4/Figs/scrna/}{Chapter4/Figs/scmet/}{Chapter4/Figs/scnmt/}}

\section{Model description}

In MOFA+ we introduce two key novelties, both in the model aspect and in the inference scheme. In the model side we introduce a principled approach for modelling multi-omic data set where the samples are structured into non-overlapping groups, where groups typically correspond to batches, donors or experimental conditions. In the inference side we implement a stochastic inference algorithm to improve scalability and enable inference with large single-cell data sets.

Formally, we generalise the model to a disjoint set of $M$ input views (i.e. groups of features) and $G$ input groups (i.e. groups of samples). The data is factorised according to the following model:
\begin{equation} \label{mofa_master_equation}
	\mathbf{Y}^{m}_{g} = \mathbf{Z}_{g} (\mathbf{W}^{m})^T + \bepsilon^{m}_{g}
\end{equation}
where $\bfZ_{g} \in \R^{N_{g} \times K}$ are a set of $G$ matrices that contain the factor values for the $g$-th group and $\bfW^{m} \in \R^{D_m \times K}$ are a set of $M$ matrices that define the feature weights for the $m$-th view. $\bepsilon^{m}_{g} \in \R^{D_m}$ captures the residuals, or the noise for each feature in each group. Notice that if $G=1$ then the model simplifies to the MOFA framework presented in Chapter 3. 

It is important to get the intuition for the multi-group formulation right. In the factor analysis setting, the aim is not to capture differential changes in \textit{mean} levels between the groups but rather to exploit the covariation patterns of the features to identify which sources of variability (i.e. latent Factors) are consistently found across multiple groups and which ones are exclusively found within a single group. This is symmetric to the interpretation of the multi-view framework in MOFA v1: the absolute levels of the features are not compared across views, only the covariation patterns are of interest. To achieve this, the features are centered per view and also per group before fitting the model. \Cref{fig:mofa2_overview} summarises the MOFA+ pipeline.

As in MOFA v1, the linearity assumptions leads to an interpretable latent space that be visualised and employed for a range of downstream analyses, including clustering, inference of non-linear differentiation trajectories, denoising and feature selection, among others. The most important extension is the generalisation of the variance decomposition analysis, where a value of variance explained per view and group is obtained for every factor. For example, imagine that Factor 1 in \Cref{fig:mofa2_overview}b corresponds to cell cycle variation, the variance decomposition analysis indicates that cell cycle is a driver of cell-to-cell heterogeneity largely in views 2 and 3, but with only minor influence in view 1. Also, this effect is manifested in groups 1 and 2, but not in group 3. This simple visualisation provides a very intuitive approach to understand variability in complex experimental designs where observations are structured into multiple views and multiple groups of cells.

\begin{figure}[H]
	\centering
	\includegraphics[width=1.00\linewidth]{mofa2_overview}
	\caption[]{\textbf{Multi-Omics Factor Analysis v2 (MOFA+) provides an unsupervised framework for the integration of multi-group and multi-view (single-cell) data.}\\
	(a) Model overview: the input data consists of multiple data sets structured into M views and G groups. Views consist of non-overlapping sets of features that often represent different assays. Analogously, groups consist of non-overlapping sets of samples that often represent different conditions or experiments. Missing values are allowed in the input data. MOFA+ exploits the covariation between the features to learn a low-dimensional representation of the data ($\bfZ$) defined by $K$ latent factors that capture the global sources of molecular variability. The weights ($\bfW$) provide a measure of feature importance. Model inference is performed using GPU-accelerated stochastic variational inference. \\
	(b) The trained MOFA+ model can be queried for a range of downstream analyses: variance decomposition, inspection of feature weights, visualisation of factors and other applications such as clustering, inference of non-linear differentiation trajectories, denoising and feature selection.
	}
	\label{fig:mofa2_overview}
\end{figure}


\subsection{Model priors and likelihood}

\subsubsection{Prior on the weights}

This remains the same as in MOFA v1. We adopt a two-level sparsity prior with an Automatic Relevance Determination per factor and view, and a (reparametrised \cite{Titsias2011}) feature-wise spike-and-slab prior:
\begin{equation}
	p(\hat{w}_{dk}^m,s_{dk}^m) = \Ndist{\hat{w}_{dk}^{m}}{0, 1/\alpha_{k}^{m}}  \text{Ber}(s_{dk}^{m} \,|\,\theta_{k}^{m})
\end{equation}
with the corresponding conjugate priors for $\theta$ and $\alpha$:
\begin{align}
	p(\theta_k^m) &= \Bdist{\theta_k^m}{a_0^\theta,b_0^\theta}\\
	p(\alpha_k^m) &= \Gdist{\alpha_k^m}{a_0^\alpha, b_0^\alpha}
\end{align}

As discussed in Chapter 3, the aim of the ARD prior is to encourage sparse associations between factors and views, such that the weight vector $\bfw_{:,k}^m$ is shrunk to zero if the factor $k$ does not explain any variation in view $m$. The aim of the spike-and-slab prior is to push individual weights to zero to yield a more interpretable solution.

\subsubsection{Prior on the factors}

In MOFA v1 we adopted an isotropic Gaussian prior which assumes an unstructured latent space \textit{a priori}:
\begin{equation}
	p(z_{nk}) = \Ndist{z_{nk}}{0,1}
\end{equation}
This is the assumption that we want to break. Following the same logic as for the weights, the integration of multiple groups of samples requires a flexible prior distribution that defines the existence of non-overlapping groups, such that the model encourages sparse solutions between factors and groups. To formalise this intuition we simply need to extrapolate the sparsity prior from the weights to the factors:
\begin{align}
	p(\hat{z}_{nk}^g,s_{nk}^g) &= \mathcal{N} (\hat{z}_{nk}^g \,|\, 0, 1/\alpha_k^g)\, \text{Ber}(s_{nk}^g \,|\,\theta_k^g) \\
	p(\theta_k^g) &= \Bdist{\theta_k^g}{a_0^\theta,b_0^\theta} \\
	p(\alpha_k^g) &= \Gdist{\alpha_k^g}{a_0^\alpha, b_0^\alpha},
\end{align}
where $g$ is the index of the sample groups.
% Notice that the spike-and-slab prior is introduced for completeness but is not necessarily required, and can be disabled by fixing $\E[\theta_k^g]=1$.

\subsubsection{Prior on the noise}

The variable $\bepsilon$ captures the residuals, or the noise, which is assumed to be normally distributed and heteroskedastic. In MOFA v2 we generalise the noise to have an estimate per feature and per group. This is important to model the case where some features may be highly variable in one group but non-variable in other groups.

\begin{align}
	p(\epsilon^{m}_{g}) &= \Ndist{\epsilon^{m}_{g}}{0,(\tau^{m}_{g})^{-1}\I} \\
	p(\tau^{m}_{g}) &= \prod_{d=1}^{D_m} \Gdist{\tau^{m}_{g}}{a_0^\tau, b_0^\tau}
\end{align}
% \begin{align}
% 	p(\epsilon^{m,g}_d) &= \Ndist{\epsilon^{m,g}_d}{0,1/\tau_d^{m,g}} \\
% 	p(\tau_{d}^{m,g}) &= \Gdist{\tau_{d}^{m,g}}{a_0^{\tau}, b_0^{\tau}}
% \end{align}

In addition, as in MOFA v1, non-Gaussian noise models can also be defined, but unless otherwise stated, we will always assume Gaussian residuals.

% \subsubsection{Likelihood}
% Altogether, this results in the following likelihood:
% \begin{equation}
% 	p(\bfY|\bfW,\bfZ,\bTau) = \prod_{m=1}^{M} \prod_{g=1}^{G} \Ndist{\mathbf{Y}^{m}_{g}}{\mathbf{Z}_{g} (\mathbf{W}^{m})^T,(\btau^{m}_{g})^{-1} \I}
% 	% p(\bfY|\bfW,\bfZ,\bTau) = \prod_{m=1}^{M} \prod_{g=1}^{G} \prod_{d=1}^{D_m} \prod_{n=1}^{N} \Ndist{y_{nd}^{mg}}{\bfz_{ng}^T\bfw_{d}^{mg},1/\tau_{d}^{mg}}
% 	% p(y_{nd}^m) = \Ndist{y_{nd}^m}{\bfz_{n,:}\bfw_{d,:}^{mT},1/\tau_d^m},
% \end{equation}

\subsubsection{Graphical model}

In summary, the updated model formulation introduces symmetric two-level sparsity priors in both the weights and the factors. The corresponding graphical model is shown below:
\begin{figure}[H]
	\centering	
	% % \definecolor{colD}{rgb}{0.2, 0.2, 0.6}
\definecolor{colM}{rgb}{0.0, 0.5, 0.0}
\definecolor{colN}{rgb}{0.5, 0.0, 0.13}
\definecolor{colG}{rgb}{1.0, 0.65, 0.0}
\newcommand\op{0.25}
\colorlet{shadecolor}{black!25}


% \begin{tikzpicture}
%   % Define nodes:
%   % matrix factorisation level
%   \node[obs]   (Y) {$y_{n,d}^m$};
%   \node[latent, above=of Y, xshift=-1.5cm] (Z) {$z_{n,k}$};
%   \node[latent, above=of Y, xshift=1.5cm] (W) {$w_{k,d}^m$};
%   \node[latent, xshift=1.5cm] (Tau) {$\tau_{d}^m$};

%   % parents of Z
%   \node[det, above=of Z] (crossZ) {$\times$};
%   \node[latent, above=of crossZ] (Zhat) {$\hat{z}_{n,k}^{\ }$};
%   \node[latent, above=of Zhat] (SigmaZ) {$\alpha_k$};
%   \node[latent, above=of crossZ, xshift=-1.5cm] (SZ) {$s_{n,k}$};
%   \node[latent, above=of SZ] (ThetaZ) {$\theta_{k}$};

%   % parents of W
%   \node[det, above=of W] (crossW) {$\times$};
%   \node[latent, above=of crossW] (What) {$\hat{w}_{k,d}^m$};
%   \node[latent, above=of What] (SigmaW) {$\alpha_k^m$};
%   \node[latent, above=of crossW, xshift=1.5cm] (SW) {$s_{k,d}^m$};
%   \node[latent, above=of SW] (ThetaW) {$\theta_{k}^m$};

%   % Connect the nodes
%   \edge {Z,W, Tau} {Y}; %
%   \edge {ThetaZ} {SZ};
%   \edge {SigmaZ} {Zhat};
%   \edge {ThetaW} {SW};
%   \edge {SigmaW} {What};
%   \factoredge {SZ, Zhat} {crossZ} {Z};
%   \factoredge {SW, What} {crossW} {W};

%   % Plates
%   \plate[] {plateK} {(Z)(W)(SZ)(Zhat)(SW)(What)(ThetaZ)(ThetaW)} {$K$};
%   \plate[] {plateN} {(Y)(Z)(crossZ)(Zhat)(SZ)(plateK.south west)} {$N$};
%   \plate[] {plateD} {(Y)(W)(Tau)(crossW)(What)(SW)(plateK.south east) (plateN.south east)} {$D_m$};
%   \plate[] {plateM} {(plateD)(ThetaW)(plateK.north east)} {$M$};

% \end{tikzpicture}






% model-binomial-probit.tex
%

% Augmented Binomial Probit regression model

%\beginpgfgraphicnamed{model-binomial-probit}

\begin{tikzpicture}
  % Define nodes:
  % matrix factorisation level
  \node[obs]   (Y) {$y_{n,d}^{g,m}$};
  \node[latent, above=of Y, xshift=-1.5cm] (Z) {$z^g_{n,k}$};
  \node[latent, above=of Y, xshift=1.5cm] (W) {$w_{k,d}^m$};
  \node[latent, yshift=-2cm] (Tau) {$\tau_{d}^{g,m}$};
  % \node[latent, densely dashed, yshift=-1.5cm] (TauN) {$\tau_{n}^{m,g}$};

  % \node[opacity=0.4,latent, xshift=-1.5cm] (Tau2) {$\tau_{n}$};

  % parents of Z
  \node[det, above=of Z] (crossZ) {$\times$};
  \node[latent, above=of crossZ] (Zhat) {$\hat{z}^g_{n,k}$};
  \node[latent, above=of Zhat] (SigmaZ) {$\alpha^g_k$};
  \node[latent, above=of crossZ, xshift=-1.5cm] (SZ) {$s^g_{n,k}$};
  \node[latent, above=of SZ] (ThetaZ) {$\theta^g_{k}$};

  % parents of W
  \node[det, above=of W] (crossW) {$\times$};
  \node[latent, above=of crossW] (What) {$\hat{w}_{k,d}^m$};
  \node[latent, above=of What] (SigmaW) {$\alpha_k^m$};
  \node[latent, above=of crossW, xshift=1.5cm] (SW) {$s_{k,d}^m$};
  \node[latent, above=of SW] (ThetaW) {$\theta_{k}^m$};

  % Connect the nodes
  \edge {Z,W} {Y};
  \edge {Tau} {Y};
  \edge {ThetaZ} {SZ};
  \edge {SigmaZ} {Zhat};
  \edge {ThetaW} {SW};
  \edge {SigmaW} {What};
  % \edge[densely dashed] {TauN} {Y}

  \factoredge {SZ} {crossZ} {Z};
  \factoredge {Zhat} {crossZ} {Z};
  \factoredge {SW} {crossW} {W};
  \factoredge {What} {crossW} {W};


  % Plates
  \plate[] {plateK} {(Z)(W)(SZ)(Zhat)(SW)(What)(ThetaZ)(ThetaW)} {$K$};
  
  % \plate[color=yellow] {plateN} {(Y)(Z)(crossZ)(Zhat)(SZ)(plateK.south west)} {$N_g$};
  \plate[] {plateN} {(Y)(Z)(crossZ)(Zhat)(SZ)(plateK.south west)} {$N_g$};
    
   % \plate[color=red] {plateG} {(plateN)(Tau)(plateK.north west)(plateN.south west)} {$G$};
   \plate[] {plateG} {(plateN)(Tau)(plateK.north west)(plateN.south west)} {$G$};
      
  
  % \plate[color=green] {plateD} {(Tau)(Y)(W)(crossW)(What)(SW)(plateK.south east)} {$D_m$};
  \plate[] {plateD} {(Tau)(Y)(W)(crossW)(What)(SW)(plateK.south east)} {$D_m$};
  
  % \plate[color=blue] {plateM} {(plateD)(ThetaW)(plateK.north east)(plateG.north east)} {$M$};
  \plate[] {plateM} {(plateD)(ThetaW)(plateK.north east)(plateG.north east)} {$M$};
  
    
    
\end{tikzpicture}

	\caption{\textbf{Graphical model for MOFA+.}\\
	The white circles represent hidden variables that are inferred by the model, whereas the grey circles represent the observed variables. There are a total of five plates, each one representing a dimension of the model: $M$ for the number of views, $G$ for the number of groups, $K$ for the number of factors, $D_m$ for the number of features in the $m$-th view and $N_g$ for the number of samples in the $g$-th group.
	}
	\label{fig:MOFA2}
\end{figure}

\subsubsection{Guidelines on the definition of views and groups} \label{section:mofa2_guidelines_views_groups}

\textbf{Views}: views typically correspond to different molecular layers, but there is flexibility in their definition and the user can explore different definitions of views. For example, one could divide the RNA expression data into three views corresponding to mRNA, rRNA and miRNA. Similarly, one can quantify DNA methylation and chromatin accessibility data over different genomic context (enhancers, promoters, etc.).

\textbf{Groups}: groups are generally motivated by the experimental design, but the user can also explore data-driven formulations. There is no \textit{right} or \textit{wrong} definition of groups, but some definitions will be more useful than others, depending on the hypothesis that is sought to explore.

\subsubsection{Model selection} \label{section:mofa2_model_selection}

As discussed in \Cref{section:mofa_robustness}, the inference procedure depends on the parameter initialisation. When using random initialisation, the Factors can vary between different model instances and a model selection step is advised. I realised that this was not a user-friendly solution and it requires a lot of computational resources when applying the model to large data sets. To simplify model training in MOFA+ we initialise the Factors using the principal components from the concatenated data set. In practice, we observe faster convergence times and better ELBO estimates when initialising with the PCA solution (\Cref{fig:mofa2_init}).

\begin{figure}[H]
	\centering
	\includegraphics[width=0.85\linewidth]{mofa2_init}
	\caption[]{
	\textbf{Comparison of PCA and Random initialisation in MOFA}.\\ Data was simulated from the generative model with the following dimensions: $M=2$ modalities, $G=2$ groups, $D=1000$ features, $N=1000$ samples and $K=10$ factors. The dashed lines mark the iteration at which the model converged.
	}
	\label{fig:mofa2_init}
\end{figure}

% COPIED
% \subsection{Solving the rotational invariance problem} \label{section:mofa2_rotational_invariance}

% Conventional Factor Analysis is invariant to rotations in the latent space \cite{Zhao2009}. To demonstrate this property, let us apply an arbitrary rotation to the weights and the factors, specified by the rotation matrix $\bfR \in \R^{K \times K}$:
% \begin{align*}
% 	\tilde{\bfZ} &= \bfZ \bfR^{-1} \\
% 	\tilde{\bfW} &= \bfR \bfW
% \end{align*}
% First, note that the model likelihood is unchanged by this rotation, irrespective of the prior distribution used.
% \begin{equation*}
% 		p(\bfY | \tilde{\bfZ} \tilde{\bfW}, \tau) = p(\bfY | \bfZ \bfR^{-1} \bfR \bfW, \tau) = p(\bfY | \bfZ \bfW, \tau)
% \end{equation*}
% However, the prior distributions of the factors and the weights are only invariant to rotations when using isotropic Normal priors:
% \begin{equation*}
% 	\ln p(\bfW) \propto \sum_{k=1}^{K} \sum_{d=1}^{D} w_{d,k}^2 = \mathrm{Tr}(\bfW^T \bfW) = \mathrm{Tr}(\bfW^T \bfR^{-1} \bfR \bfW) = \mathrm{Tr}(\tilde{\bfW^T} \tilde{\bfW})
% \end{equation*}
% where we have used the property $\bfR^{T} = \bfR^{-1}$ that applies to rotation matrices. The same derivation follows for the factors $\bfZ$.\\
% In practice, this property renders conventional Factor Analysis unidentifiable, hence limiting its interpretation and applicability. Sparsity assumptions, however, partially address the rotational invariance problem \cite{Hore2015}.

% It is important to remark that the factors are nonetheless invariant to permutations. This implies that under different initial conditions, the order of the factors is not necessarily the same in independent model fittings. To address this we manually sort factors \textit{a posteriori} based on total variance explained.


% \subsection{Stochastic variational inference algorithm}

% In \Cref{section:stochastic_variational_inference} I have explained how to derive a stochastic variational inference (SVI) algorithm for a general Bayesian model using an adapted version of the algorithm introduced in \cite{Hoffman2012}.\\
% To apply the SVI algorithm to MOFA the first step is to choose the \textit{local} and \textit{global} dimensions such that the \textit{local} dimension will be factorised in the ELBO and where the stochastic gradients apply. 

% In single-cell studies we expect increasingly large data sets in the number of cells, but the number of features are expected to remain roughly constant. Thus, the natural dimension to define as \textit{local} is the axis of samples. In the case of MOFA+, the variables that are classified as \textit{local} are the Factors $\mathbf{Z_g} = \{ z_{nk}^g \}$, which due to the reparametrisation of the spike-and-slab prior consists on the element-wise product of two matrices: $\hat{\mathbf{Z_g}}$ and $\mathbf{S_g}$. All other hidden variables are global. 

% This leads to the following SVI algorithm:
% \begin{algorithm}[h!]
%   \caption{Stochastic mean-field variational inference for MOFA+}
%   \begin{algorithmic}[1]
% 	\State Initialise randomly the parameters of the global variables \{$\bf\tau^{gm}$, $\bf\hat{W}^m$, $\bfS^\bfm$, $\bf\alpha^m, \bf\theta^m$, $\bf\alpha^g$, $\bf\theta^g$\}.
% 	\State Initialise the step size $\rho^{(t=0)}$
% 	\Repeat
% 	    \State \text{sample $\mathcal{B}$ a mini-batch of samples of size $S << N$}
% 		\For{\text{each local variational parameter $\phi_{nk}^g$ of nodes \{$\hat{z_{nk}^g}$, $s_{nk}^g$\} such that $n$ is in batch $\mathcal{B}$}} \\
% 			\State $\phi_{nk}^{(t+1)}$ is the updated parameter $\phi_{nk}$ following the classic VI update equation \\
%       	\EndFor
% 		\For{\text{each global variational parameter $\lambda$ of nodes \{$\bf\tau^{gm}$, $\bf\hat{\bfW}^m$, $\bfS^m$, $\bf\alpha^m, \bf\theta^m$, $\bf\alpha^g$, $\bf\theta^g$\}}}
% 		     \State
%         		\begin{align} \label{eq_elbo_factorised} \begin{split}
%             	\lambda^{(t + 1)} &= (1-\rho ^{(t)})\lambda^{(t)} +  \rho ^{(t)} \lambda_{\mathcal{B}}^{(t+1)}
%             \end{split} \end{align}
%             \State \text{where $\lambda_{\mathcal{B}}^{(t+1)}$ is the updated parameter $\lambda$ following the classic VI update equation,}
%             \State \text{but considering the selected batch $\mathcal{B}$ repeated $N/S$ times instead of the full dataset.}
%       	\EndFor
% 	\Until{ELBO convergence}
% 	\end{algorithmic}
% 	\label{MOFAstochasticascent}
% \end{algorithm}

% \subsection{Theoretical comparison with published methods}
% As discussed in Chapter 3, a variety of factor analysis models exist with the aim of perfoming multi-view and/or multi-group data integration. A comparison of multi-view method is shown in \Cref{GFAtable}. Here we specifically focus on the methods that have been applied to single-cell data:
% \begin{table}[h]
% 	\begin{tabular}{@{}llllll} 
% 		\toprule
% 		{\textbf{Method}} & {\textbf{\parbox{2.0cm}{Scales to\\$>1e5$ cells?}}} & {\textbf{\parbox{2.0cm}{Multi-view}}} & {\textbf{\parbox{2.0cm}{Multi-group}}} & {\textbf{\parbox{2.0cm}{Missing values}}} & {\textbf{\parbox{2.0cm}{Likelihoods}}} \\ \toprule
% 		slalom \cite{Buettner2017} & No & No & No & No & ZI gaussian \\\midrule
% 		pCMF \cite{Durif2019} & Yes & No & No & No & ZI poisson \\\midrule
% 		ZIFA \cite{Pierson2015} & No & No & No & No & ZI gaussian \\\midrule
% 		scVI \cite{Lopez2018} & Yes & No & Yes & No & ZI negative binomial \\\midrule
% 		MSFA \cite{DeVito2019} & No & No & Yes & No & Gaussian \\\midrule
% 		ZINB-WaVE \cite{Risso2018} & No & No & No & No & ZI negative binomial \\\midrule
% 		AJIVE \cite{Feng2018} & No & Yes & No & No & NA \\\midrule
% 		DIABLO \cite{Singh2018} & Yes & Yes & No & * & NA \\\midrule
% 		scHPF \cite{Levitin2019} & Yes & No & No & Yes & Negative binomial \\\midrule
% 		MOFA \cite{Argelaguet2018} & No & Yes & No & Yes & Gaussian/Poisson/Bernoulli \\\midrule
% 		MOFA+ & Yes & Yes & Yes & Yes & Gaussian/Poisson/Bernoulli \\\midrule
% 	\end{tabular}
% 	\caption{Overview of Factor analysis methods for single-cell data.}
% 	% \label{XXX}
% \end{table}


\subsection{A note on the implementation}

The core of MOFA+ is implemented in Python, and the downstream analysis and visualisations are implemented in R. GPU acceleration is implemented using CuPy \cite{Okuta2017}, an open-source matrix library accelerated with NVIDIA CUDA. To facilitate adoption of the method, we deployed MOFA+ as open-source software\footnote{\url{https://github.com/bioFAM/MOFA2}} with multiple tutorials and a web-based analysis workbench\footnote{\url{http://www.ebi.ac.uk/shiny/mofa/}}.%, hopefully enabling a user-friendly exploration of complex single-cell data sets.



