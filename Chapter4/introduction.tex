\graphicspath{{Chapter4/Figs/simulations/}{Chapter4/Figs/scrna/}{Chapter4/Figs/scmet/}{Chapter4/Figs/scnmt/}}

\chapter{MOFA+: an improved framework for the integration of structured single-cell data}

In Chapter 2 we developed Multi-Omics Factor Analysis (MOFA), a statistical framework for the unsupervised integration of multi-omics data. \\
MOFA addresses key challenges in data integration, including overfitting, noise reduction, handling of missing values and improved interpretation of the model output. However, when applied to increasingly-large data sets, as in the single-cell domain, the inference scheme implemented in MOFA is still limited in scalability. In addition, the increased experimental throughput has facilitated the simultaneous study of multiple experimental conditions, even in a combinatorial fashion\cite{Replogle2020}. By pooling and contrasting information across experimental conditions, it would be possible to obtain more comprehensive insights into the complexity underlying biological systems.

MOFA makes strong assumptions about the dependencies across samples and it offers no principled way of modelling data sets where the samples are structured into multiple groups, where groups can correspond to batches, donors or independent studies. In this new Chapter we improve the model formulation in MOFA with the aim of performing integrative analysis of large-scale datasets where the features are structured into multiple data modalities and/or the samples (or cells) are also structured into different groups.

The work discussed in this chapter has been peer-reviewed and published in \cite{Argelaguet2020}. The project was conceived by Damien Arnol, Britta Velten and me. The mathematical derivations and the implementation of the stochastic variational inference scheme was done by Damien Arnol, Yonatan Deloro and me. The downstream analysis package was implemented by Danila Bredikhin and me. I generated most figures and I wrote the manuscript with feedback from all authors. John C. Marioni and Oliver Stegle supervised the project. 

\section{Theoretical foundations}

\subsection{Exponential family distributions} \label{section:exponential_family}

Exponential family distributions are a parametric class of probability distributions that have characteristic mathematical properties which make them amenable for probabilistic modelling.\\
The majority of probability distributions that are commonly used in statistics belong to the exponential family, including the normal or Gaussian, Gamma, Poisson, Bernoulli, Exponential, etc.\\
Formally, exponential family distributions can be represented in the following form:
\begin{equation} \label{eq:exponential_family}
	p(\bfx|\btheta) = h(\bfx) \exp \{ \eta(\btheta) T(\bfx) - A(\btheta) \}
\end{equation}
where $\bfx$ is a multivariate random variable and $\btheta$ are the distribution's parameters. Each term has a common notation: T(\bfx): sufficient statistics; $\eta(\btheta)$: natural parameters; $h(\bfx)$: base measure; $A(\eta)$: the log-partition function (or the normaliser).

The exponential family form for the probability distributions tht are frequently used in this thesis are shown below:

Univariate normal distribution:
\begin{align*}
	& \eta(\mu,\sigma) = \lbrack \frac{\mu}{\sigma^2}; -\frac{1}{2\sigma^2} \rbrack \\
	& h(x) = \frac{1}{\sqrt{2\pi}} \\
	& T(x) = \lbrack x; x^2 \rbrack \\
	& A(\mu,\sigma) = \frac{\mu^2}{2\sigma^2} + \log \| \sigma \|
\end{align*}

Multivariate normal distribution:
\begin{align*}
	& \eta(\bmu,\bSigma)  = \lbrack\Sigma^{-1} \mu; -0.5\Sigma^{-1} \rbrack \\
	& T(x) = \lbrack x; xx^T \rbrack \\
	& h(x) = (2\pi)^{-\frac{k}{2}} \\
	& A(\theta) = -0.25\eta_1^T \eta_2{-1} \eta_1 - 0.5\log(\|-2\eta_2\|)
\end{align*}

Gamma distribution:
\begin{align*}
	& \eta = \lbrack \alpha - 1; -\beta \rbrack \\
	& T(x) = \lbrack \log x; x \rbrack \\
	& h(x) = 1 \\
	& A(\theta) = \log(\Gamma(\eta_1 + 1)) - (\eta_1 + 1) \log(-\eta_2)
\end{align*}

Beta distribution:
\begin{align*}
	& \eta = [\alpha; \beta] \\
	& T(x) = [\log x; \log (1-x)] \\
	& h(x) = \frac{1}{x(1-x)} \\
	& A(\theta) = \log(\Gamma(\eta_1)) +\log(\Gamma(\eta_2)) - \log(\Gamma(\eta_1+\eta_2))
\end{align*}

In the context of Bayesian inference, the main property that make exponential family distributions indispensable is that they have conjugate priors (i.e. a combination of likelihood and prior distributions which ensure a closed-form posterior distribution which is of the same form as the prior). As we have discussed in Chapter 2, this property is essential for enabling efficient statistical inference, otherwise posterior distributions must be computed using expensive and approximate numerical methods.


\subsection{Gradient ascent} \label{section:gradient_ascent}

Gradient ascent is a first-order optimization algorithm for finding the maximum of a function \cite{Bishop2006,Murphy}. Formally, for a differentiable function $F(x)$, the iterative scheme of gradient ascent is:
\begin{equation} \label{gradient_ascent}
	\bfx^{(t+1)} = \bfx^{(t)} + \rho^{(t)} \nabla F(\bfx^{(t)})
\end{equation}
In short, the algorithm works by taking steps proportional to the gradient $\nabla F$ evaluated at each iteration $t$. 
% This leads to a monotonic sequence:
% \[
% 	\bfx^{0} \leq \bfx^{1} \leq \bfx^{1} \cdots 
% \]
Importantly, the step size $\rho^{(t)}$ is typically adjusted at each iteration $t$ such that it satisfies the Robbins-Monro conditions: $\sum_t \rho^{(t)} = \infty \text{ and } \sum_t (\rho^{(t)})^2 < \infty$. $F$ is guaranteed to converge to the global maximum \cite{Robbins-Monro1951} if the objective function is convex. If $F$ is not convex, the algorithm is sensible to the initialisation $\bfx^{t=0}$ and can converge to local optima.


\subsubsection{Stochastic gradient ascent} \label{section:stochastic_gradient_ascent}

Gradient ascent becomes prohibitively slow with large datasets, mainly because of the computational cost involved in the iterative calculation of gradients \cite{Spall2003}.\\
A simple strategy to speed up gradient ascent is to replace the actual gradient $\nabla F$ by an estimate $\hat{\nabla} F$ using a random subset of the data (minibatch).
The iterative scheme is then defined in the same way as in standard gradient ascent:
\begin{equation}
	\bfx^{(t+1)} = \bfx^{(t)} + \rho^{(t)} \hat{\nabla} F(\bfx^{(t)})
\end{equation}

%In practice, the stochastic nature of the algorithm makes the optimisation trajectory more wiggly and typically requires a larger number of iterations than standard gradient ascent. However, the reduced computational cost in computing the gradients yields an overall faster training time

% Copied from hoffman
% Under the right conditions, stochastic optimization algorithms provably converge to an optimum of the objective. Stochastic optimization is particularly attractive when the objective (and therefore its gradient) is a sum of many terms that can be computed independently. In that setting, we can cheaply compute noisy gradients by subsampling only a few of these terms.


\subsubsection{Natural gradient ascent} \label{section:natural_gradient_ascent}

Gradient ascent becomes problematic when applied to probabilistic models. To give the intuition, consider a probabilistic model with a hidden variable $x$ and corresponding parameter $\theta$, with a general objective function $\Lagr(\theta)$. From the definition of a derivative:
\[
	\nabla \Lagr(\theta) = \lim_{||h||\to0} \frac{\Lagr(\theta + h) - \Lagr(\theta)}{||h||}
\]
where $h$ represents an infinitesimally small positive step in the space of $\theta$.\\
To find the direction of steepest ascent, one would need to search over all possible directions $d$ in an infinitely small distance $h$, and select the $\hat{d}$ that gives the largest gradient:
\[
\nabla \Lagr(\theta) = \lim_{h\to0} \frac{1}{h}\argmax_{d \, s.t. \|d\|=h} \Lagr(\theta+d) - \Lagr(\theta)
\]
Importantly, this operation requires a distance metric to quantify what a \textit{small} distance $h$ means. In standard gradient ascent, this is measured using an Euclidean norm, and the direction of steepest ascent is hence dependent on the Euclidean geometry of the $\theta$ space. Why is this problematic when working with probability distributions? Because it does not consider the uncertainty that underlies probability distributions. When $\theta$ is a random variable with an underlying probability distribution, a small step from $\theta^{(t)}$ to $\theta^{(t+1)}$ does not guarantee an equivalently small change from $\Lagr(\theta^{(t)})$ to $\Lagr(\theta^{(t+1)})$. To illustrate this, consider the following example of four random variables:
\begin{equation}
	\begin{split}
		\psi_1 &\sim \Ndist{0}{5} \\
		\psi_2 &\sim \Ndist{10}{5}
	\end{split}
	\qquad
	\begin{split}
		\psi_3 &\sim \Ndist{0}{1} \\
		\psi_4 &\sim \Ndist{10}{1}
	\end{split}
\end{equation}
Using the Euclidean metric, the distance between $\psi_1$ and $\psi_2$ is the same as the distance between $\psi_3$ and $\psi_4$. However, the distance in distribution space (measured for example by the KL divergence) is much larger between $\psi_1$ and $\psi_2$ than between $\psi_3$ and $\psi_4$:

\begin{figure}[!h]
	\begin{center}
		\includegraphics[width=0.65\textwidth]{mofa2_euclidean_distributions}
		\caption{\textbf{Illustration of the problem of using the Euclidean norm as a distance measure between parameters of probability distributions}.\\
		In both plots, the red and blue distributions are separated by the same Euclidean distance of 100. Yet, the distance in probability space between the two distributions is higher in the right.
		}
		\label{fig:mofa2_euclidean_distributions}
	\end{center}
\end{figure}

This basic simulation suggests that replacing the Euclidean distance by the KL divergence as a distance metric may be more appropriate in the context of probabilistic modelling:
\[
	\nabla_{KL} \Lagr(\theta) = \lim_{h\to0} \frac{1}{h}\argmax_{d \, s.t. KL[p_\theta||p_{\theta+d}]=h} \Lagr(\theta+d) - \Lagr(\theta)
\]
The direction of steepest ascent measured by the KL divergence is called the natural gradient \cite{Amari1998,Martens2014}. To find the optimal $\hat{d}_{KL}$, one needs to solve the following optimisation problem:
\begin{equation*} \begin{aligned}
	&\argmin_{d} \Lagr(\theta+d) \qquad
	& \text{subject to}
	& \quad KL[p_\theta||p_{\theta+d}] < c
\end{aligned} \end{equation*}
where $c$ is an arbitrary constant. Previous works have shown that this can be solved by introducing Lagrange multipliers and Taylor expansions\cite{Amari1998,Kristiadi2019}. The solution corresponds to the standard (Euclidean) gradient pre-multiplied by the inverse of the Fisher Information Matrix of $q(x|\theta)$:
\begin{equation}\label{natural_gradient}
	\hat{d}_{KL} \propto \bfF^{-1}(\theta) \nabla_{\theta} \Lagr(\theta)
\end{equation}
where $\bfF(\theta)$ is defined as
\[
	\bfF(\theta) = \E_{q(x|\theta)}[(\nabla_\theta \log q(x|\theta)) (\nabla_\theta \log q(x|\theta))^T]
\]
%Effectively, the premultiplication by $\bfF^{-1}$ takes into account the local curvate of $q(\theta)$ in distribution space. \\

%Importantly, when $q(x|\theta)$ belongs to the exponential family, the Fisher Information matrix is simply the Hessian of the log normalizer.\\

In conclusion, while the standard gradient points to the direction of steepest ascent in Euclidean space, the natural gradient points to the direction of steepest ascent in a space where distances are defined by the KL divergence \cite{Kristiadi2019,Amari1998,Hoffman2012}.


\subsection{Stochastic variational inference} \label{section:stochastic_variational_inference}

In this section I will demonstrate how to derive a stochastic variational inference algorithm for general Bayesian models. This work is inspired from \cite{Hoffman2012} which we adapted and implemented for the MOFA model. A comprehensive mathematical derivation of the algorithm is not sought in this chapter. Instead, I will describe a modified and simplified derivation to gist the essential. For a complete mathematical derivation we refer the reader to \cite{Hoffman2012}.\\
Also, this section builds upon three theoretical foundations that have been introduced before: Variational inference (\Cref{section:variational_inference}), exponential family distributions (\Cref{section:exponential_family}) and (natural) gradient ascent (\Cref{section:gradient_ascent}).

% COPIED
%The stochastic nature of the approach is interesting when one dimension of the matrix of observed variables is much larger than the others. In our case, it corresponds to $N$, the number of samples (or cells). \\


% Copoed from Hoffman
% In variational inference, we define a flexible family of distributions over the hidden variables, indexed by free parameters (Jordan et al., 1999; Wainwright and Jordan, 2008). We then find the setting of the parameters (i.e., the member of the family) that is closest to the posterior. Thus we solve the inference problem by solving an optimization problem.\\



\subsubsection{Model definition}

Consider a probabilistic model with a set of unobserved random variables, observations and (non-random) parameters. We begin by classifying the variables of the model into four different categories:

% what about the params for the local variables???
\begin{itemize}
	%\itemsep-1.5em
	\item observations ($\bfY$): $N$ different vectors $\bfy_{n}$, each one containing the observed variables for the $n$-th sample.
	\item local (hidden) variables ($\bfZ)$: $N$ different vectors $\bfz_{n}$, each one containing $K$ hidden variables associated with the $n$-th sample.
	\item global (hidden) variables ($\bbeta$): one vector that contains $B$ hidden variables not indexed by $n$.
	\item parameters (non-random) for the global variables ($\balpha_{\beta}$).
	\item parameters (non-random) for the local variables ($\balpha_{z}$).
\end{itemize}

This leads to the following factorisation of the joint distribution:
\begin{equation}
	p(\bfY, \bfZ, \balpha_{\beta}, \balpha_{z}) = p(\bfZ|\balpha_{z}) p(\bbeta|\balpha_{\beta}) \prod_{n=1}^{N} p(\bfy_{n}|\bfz_{n},\bbeta)
\end{equation}
and the corresponding graphical model representation:
\begin{figure}[H]
	\centering	
	\begin{tikzpicture}

% Define nodes
\node[obs]   (Y) {$y_{n,d}$};
\node[latent, above=of Y, xshift=-1.5cm] (Z) {$z_{n,k}$};
\node[latent, above=of Y, xshift=1.5cm] (W) {$w_{d,k}$};
\node[latent, xshift=1.5cm] (Tau) {$\tau_{d}$};

% Connect the nodes
\edge {Z,W, Tau} {Y};

% Plates
\plate[] {plateK} {(Z)(W)} {$K$};
\plate[] {plateN} {(Y)(Z)(plateK.north west)} {$N$};
\plate[] {plateD} {(Y)(W)(Tau)(plateK.south east) (plateN.south east) (plateN.north east)} {$D$};

\end{tikzpicture}
	\caption{\textbf{Graphical model for a general probabilistic model where unobserved variables are classified as global and local.}\\
	The dashed line indicates that the connection between global and local variables is optional and it is not used in the MOFA model.
	}
	\label{fig:graphical_model_stochastic}
\end{figure}

Notice that the difference between local and global variables lies on the conditional dependency assumptions. The local variables for the $n$-th sample $\bfz_n$ are conditionally independent from any other observation $\bfy_{j}$ or local variable $\bfz_{j}$ (where $j \neq n$), given that the global variables $\beta$ are observed:
\[
	p(\bfy_n,\bfz_n| \bfy_j,\bfz_{nj},\bbeta,\balpha_{z_{n}},\balpha_{z_{j}}) = p(\bfy_n,\bfz_n|\bbeta,\balpha_{z_{n}})
\]
To relate this formulation to the MOFA model, the local variables would contain the factors whereas the global variables would contain the feature weights.

For simplicity in the derivation, we will assume the existence of a single global variable $\beta$, a single parameter $\alpha_{\beta}$ for the global variables and a single parameter $\alpha_{z_{nk}}$ for each local variable.

The first assumption in the model is that the prior distributions of the local and global variables are members of the exponential family (see \Cref{eq:exponential_family})
\begin{align} \label{eq_priors} 
	\begin{split}
	p(\beta|\alpha_{\beta}) = h(\beta) \exp\{ \eta_g(\alpha_{\beta}) t(\beta) - a_g(\alpha_{\beta}) \} \\
	p(z_{nk}|\alpha_{z}) = h(z_{nk}) \exp\{ \eta_l(\alpha_{z}) t(z_{nk}) - a_l(\alpha_{z}) \}
	\end{split} 
\end{align}

The second assumption is that the complete conditionals of the unobserved variables are also members of the exponential family:
\begin{align} \label{eq_complete_conditionals} 
	\begin{split}
	p(\beta|\bfY,\bfZ,\balpha) = h(\beta) \exp\{ \eta_g(\bfY,\bfZ,\balpha)^T t(\beta) - a_g(\eta_g(\bfY,\bfZ,\balpha)) \} \\
	p(\bfz_{n}|\bfy_{nj},\bfz_{nj},\beta) = h(\bfz_{n}) \exp\{ \eta_l(\bfy_{nj}, \bfz_{nj},\beta)^T t(\bfz_{n}) - a_l(\eta_l(\bfy_{nj},\bfz_{nj},\beta)) \}
	\end{split} 
\end{align}

\subsubsection{Setting up the inference problem}

 First, we set up the variational distributions for both the local variables and the global variables. Here we are going to assume that all unobserved variables are independent (mean-field assumption)
\[
	q(\bfz,\beta) = q(\beta|\lambda) \prod_{n=1}^{N} \prod_{k=1}^{K} p(z_{nk}|\phi_{nk})
\]
and belong to the same exponential family as the corresponding prior distribution:
\begin{align} \label{eq_variational_distributions}
	q(\beta|\lambda) &= h(\beta) \exp\{ \eta_g(\lambda) t(\beta) - a_g(\lambda) \} \\
	q(z_{nk}|\phi_{nk}) &= h(z_{nk}) \exp \{ \eta_l(\phi_{n}) t(z_{nk}) - a_l(z_{nk}) \}
\end{align}
where $\lambda$ are the parameters governing the variational distribution for the global variables and $\phi_{nk}$ are the parameters governing the variational distribution for the $k$-th local variable and the $n$-th sample.

From the assumptions above, the ELBO (the objective function in variational inference, see \Cref{eq:XXX}) factorises as:
\begin{align} \label{eq_elbo_factorised} \begin{split}
	\Lagr &= \E_{q(\bfZ,\beta)}[\log p(\bfY,\bfZ,\beta)] - \E_{q(\bfZ)}[\log q(\bfZ)] - \E_{q(\beta)}[\log q(\beta)] \\
	 &= \sum_{n=1}^{N} \E_{q(\bfz_n,\beta)}[\log p(\bfy_n,\bfz_n,\beta)] - \sum_{n=1}^{N} \sum_{k=1}^{K} \E_{q(z_{nk})}[\log q(z_{nk})] - \E_{q(\beta)}[\log q(\beta)]
\end{split} \end{align}

Notice that the objective decomposes into global terms (not involving $N$) and local terms (involving $N$). Importantly, the local terms can be approximated using estimates of the gradient by subsampling the data set. Assumign a mini-batch of size $S$:
\[
	\hat{\Lagr} = \frac{N}{S} \sum_{n=1}^{S} \E_{q(\bfz_n,\beta)}[\log p(\bfy_n,\bfz_n,\beta)] - \frac{N}{S}\sum_{s=1}^{S} \sum_{k=1}^{K} \E_{q(z_{nk})}[\log q(z_{nk})] - \E_{q(\beta)}[\log q(\beta)]
\]
If the samples are independent then the expectation of this noisy gradient is equal to the true gradient. This is the main principle of stochastic optimisation. The next step is to derive an iterative algorithm to find the values of the variational parameters that maximise the ELBO.


\subsubsection{Calculating the gradient for the global parameters}

To derive the updates for the global parameters we first write the ELBO in terms of $\lambda$:
\[
	\Lagr(\lambda) = \E_{q(z,\beta)}[\log p(\beta|\bfY,\bfZ)] - \E_{q(\beta)}[\log q(\beta)] + \const
\]
where the constant term captures all quantities that do not depend on $\beta$. Then, from the assumption that the complete conditionals and the variational distributions belong to the exponential family (\Crefrange{eq_complete_conditionals}{eq_variational_distributions}):
\baln
	\Lagr(\lambda) &= \E_{q(z,\beta)}[\eta_g(\bfY,\bfZ,\balpha)^T t(\beta)] - \E_{q(\beta)}[\lambda^T t(\beta) - a_g(\lambda) ] + \const \\
	&= \E_{q(z)}[\eta_g(\bfY,\bfZ,\balpha)^T] \nabla a(\lambda) - \lambda^T \nabla a_g(\lambda) - a_g(\lambda) + \const
\ealn
where we have used the exponential family identity $\E_{q(\beta)}[t(\beta)] = \nabla a_g(\lambda)$.

Taking the gradient with respect to $\lambda$:
\begin{equation} \label{gradient_global}
	\nabla_{\lambda} \Lagr(\lambda) = \nabla_{\lambda}^{2} a_g(\lambda)(\E_{q(z)}[\eta_g(\bfY,\bfZ,\balpha)] - \lambda)
\end{equation}
and setting it to zero leads to the solution:
\begin{equation} \label{solution_global}
	\lambda = \E_{q(z)}[\eta_g(\bfY,\bfZ,\balpha)]
\end{equation}


\subsubsection{Calculating the gradient for the local parameters}

Turning to the local parameters, as a function of $\phi_{nk}$ the ELBO becomes:
\[
	\Lagr(\phi_{nk}) = \E_{q(\beta,\bfz_{nj})}[\log p(\bfz_{nj}|\bfy_{n},\bfz_{nj}, \beta)] - \E_{q(z_{nk})}[\log q(z_{nk})] + \const
\]
Again, from the assumption that the complete conditionals and the variational distributions belong to the exponential family (\Crefrange{eq_complete_conditionals}{eq_variational_distributions}):
\begin{align*}
	\Lagr(\phi_{nk}) &= \E_{q(\beta,\bfz_{nj})}[\eta_l(\bfy_n,\bfz_{nj},\beta)^T t(\bfz_{nj})] - \E_{q(z_{nk})}[\phi_{nk} t(z_{nk}) - a_l(\phi_{nk}) ] + \const \\
	&= \E_{q(\beta,\bfz_{nj})}[\eta_l(\bfy_n,\bfz_{nj},\beta)]^T \nabla a_l(\phi_{nk}) - \phi_{nk} \nabla a_l(\phi_{nk}) - a_l(\phi_{nk}) + \const
\end{align*}

Taking the gradient with respect to $\phi_{nk}$:
\begin{equation} \label{gradient_local}
	\nabla_{\phi} \Lagr(\phi_{nk}) = \nabla_{\phi}^2 a_l(\phi_{nk}) (\E_{q(\beta,\bfz_{nj})}[\eta_l(\bfy_n,\bfz_{nj},\beta)] - \phi_{nk})
\end{equation}

and setting it to zero leads to the following solution:
\begin{equation} \label{solution_local}
	\phi_{nk} = \E_{q(\beta,\bfz_{nj})}[\eta_l(\bfy_{n},\bfz_{nj},\beta)]
\end{equation}


\subsubsection{Coordinate ascent variational inference algorithm}

Now that we have the gradients for both the local and the global parameters, we can define a gradient ascent algorithm to optimise the model:

\begin{algorithm}
  \caption{Coordinate ascent variational inference algorithm}
  \begin{algorithmic}[1]
	\State Initialise the global parameters $\blambda^{(t=0)}$
	\Repeat
		\For{\text{each local variational parameter $\phi_{nk}$}}
			\State $ \phi_{nk}^{(t+1)} \gets \E_{q^{(t)}}(q(\bbeta,\bfz_{nj}))[\eta_l(\bfy_{n},\bfz_{nj},\bbeta)] $
      	\EndFor
		\For{\text{each global variational parameter $\lambda$}}
			\State $ \lambda{(t+1)}= \E_{q^{(t)}}(q(z))[\eta_g(\bfY,\bfZ,\balpha)] $
      	\EndFor
	\Until{Convergence}
	\end{algorithmic}
\end{algorithm}

However, as discussed in \Cref{section:natural_gradient_ascent}, the use of euclidean-based gradients ignores important information about the geometry of the distribution and is thus not optimal for the optimisation of probabilistic models. Next, we will derive a similar coordinate ascent algorithm but using instead the natural gradient.


\subsubsection{Deriving the natural gradients for the global variational parameters}

From \Cref{gradient_global}, the gradient of the ELBO with respect to the global parameters $\lambda$ is:
\[
	\nabla_{\lambda} \Lagr(\lambda) = \nabla_{\lambda}^{2} a_g(\lambda)(\E_{q(z)}[\eta_g(\bfY,\bfZ,\balpha)] - \lambda)
\]
Premultiplying by $\bfF(\beta)^{-1}=\nabla_{\lambda}^{2} a_g(\lambda)$ gives the natural gradient for the global parameters:
\[
	\hat{\nabla}_{\lambda} \Lagr(\lambda) = \E_{q(z)}[\eta_g(\bfY,\bfZ,\balpha)] - \lambda
\]


\subsubsection{Deriving the natural gradients for the local variational parameters}

From \Cref{gradient_local}, the gradient of the ELBO with respect to the local parameters $\bphi$ is:
\[
	\nabla_{\phi} \Lagr(\phi_{nk}) = \nabla_{\phi}^2 a_l(\phi_{nk}) (\E_{q(\beta,\bfz_{nj})}[\eta_l(\bfy_n,\bfz_{nj},\beta)] - \phi_{nk})
\]
Premultiplying by $\bfF(z_{nk})^{-1}=\nabla_{\phi}^{2} a_l(\phi_{nk})$ gives the natural gradient for the global parameters:
\[
	\hat{\nabla}_{\phi} \Lagr(\phi_{nk}) = \E_{q(\beta,\bfz_{nj})}[\eta_l(\bfy_n,\bfz_{nj},\beta)] - \phi_{nk}
\]

Remarkably, the natural gradient for both the local and global variational parameters is simply the standard gradient subtracting the current value of the parameters. Thus, the Fisher Information matrix does \textit{not} need to be explicitly computed at each iteration, which leads to a considerable simplification of the problem. 


\subsubsection{Stochastic variational inference algorithm using natural gradients}

After replacing the euclidean gradient with the natural gradients, the model can be optimised using the following algorithm: 

\begin{algorithm}[H]
  \caption{Stochastic variational inference algorithm using natural gradients}
  \begin{algorithmic}[1]
	\State Initialise the global parameters $\blambda^{(t=0)}$.
	\State Initialise step size $\rho^{(t=0)}$
	\Repeat
	    \State \text{sample $\mathcal{B}$ a mini-batch of samples of size $S$}
		\For{\text{each local variational parameter $\phi_{nk}$ such that $n$ is in batch $\mathcal{B}$}}
			\State $$ \phi_{nk}^{(t+1)} = \E_{q^{(t)}(\bbeta,\bfz_{nj})}[\eta_l(\bfy_{n},\bfz_{nj},\bbeta)] $$
      	\EndFor
		\For{\text{each global variational parameter $\lambda$}}
		     \State
        		\begin{align} \label{eq_elbo_factorised} \begin{split}
            	\lambda^{(t + 1)} &=  \lambda^{(t)} +  \rho ^{(t)} \hat{\nabla}_{\lambda} \Lagr^S(\lambda) \\
            	 &=  (1-\rho ^{(t)})\lambda^{(t)} +  \rho ^{(t)} \E_{q^{(t+1)}(z)}\left[\frac{N}{S}\eta_g(\bfY_{[n \in \mathcal{B}],:},\bfZ_{[n \in \mathcal{B}],:},\balpha)\right]
            \end{split} \end{align}
            \State where $[n \in \mathcal{B}]$ denotes the subset of indices corresponding to the samples in $\mathcal{B}$
      	\EndFor
	\Until{Convergence}
	\end{algorithmic}
\end{algorithm}

%TO-DO: EXPLANATION AND SIMPLIFICATION

Notice that the stochastic nature of the algorithm introduces additional hyperparameters:

\begin{itemize}
    \item \textbf{Batch size}: controls the number of samples that are used to compute the gradients at each iteration. A trade off exists where high batch sizes lead to a more expensive computation of the gradient but yield a less noisy estimate.

    \item \textbf{Learning rate}: The learning rate $p(t)$ controls the step size in the direction of the natural gradient, with high learning rates leading to higher steps. In the natural gradient setting, the learning rate also controls how much memory from previous iterations is translated to the current updates. The particular case of a constant learning rate of $1$ yields no memory from previous iterations (thus simplifies to standard gradient ascent). To ensure proper convergence, the learning rate has to be decayed during training. Several strategies exist\cite{Ranganath2013}, here we used the simple function $\rho(t) = \frac{\rho_0}{(1 + \kappa t)^{3/4}}$, which introduces two extra hyperparameters: (1) The forgetting rate $\kappa$, which controls the decay of the learning rate, and $\rho_0$ which determines the initial learning rate.

\end{itemize}