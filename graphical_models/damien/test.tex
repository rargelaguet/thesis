\documentclass[a4paper,11pt]{article}
\usepackage{amsmath}
\usepackage{tikz}
\usetikzlibrary{positioning}
\usetikzlibrary{bayesnet}


\begin{document}

% \begin{center}
% 	  \input{biofam_clust}
% \end{center}
\begin{figure}
  \begin{center}
  	  % model-binomial-probit.tex
%

% Augmented Binomial Probit regression model

%\beginpgfgraphicnamed{model-binomial-probit}

\definecolor{colD}{rgb}{0.2, 0.2, 0.6}
\definecolor{colM}{rgb}{0.0, 0.5, 0.0}
\definecolor{colN}{rgb}{0.5, 0.0, 0.13}
\definecolor{colG}{rgb}{1.0, 0.65, 0.0}
\newcommand\op{0.25}
\colorlet{shadecolor}{black!25}


\begin{tikzpicture}
  % Define nodes:
  % matrix factorisation level
  \node[obs]   (Y) {$y_{n,d}^m$};
  \node[latent, above=of Y, xshift=-1.5cm] (Z) {$z_{n,k}$};
  \node[latent, above=of Y, xshift=1.5cm] (W) {$w_{k,d}^m$};
  \node[latent, xshift=1.5cm] (Tau) {$\tau_{d}^m$};

  \node[opacity=0.4,latent, xshift=-1.5cm] (Tau2) {$\tau_{n}$};

  % parents of Z
  \node[det, above=of Z] (crossZ) {$\times$};
  \node[latent, above=of crossZ] (Zhat) {$\hat{z}_{n,k}^{\ }$};
  \node[latent, above=of Zhat] (SigmaZ) {$\alpha_k$};
  \node[latent, above=of crossZ, xshift=-1.5cm] (SZ) {$s_{n,k}$};
  \node[latent, above=of SZ] (ThetaZ) {$\theta_{k}$};

  % parents of W
  \node[det, above=of W] (crossW) {$\times$};
  \node[latent, above=of crossW] (What) {$\hat{w}_{k,d}^m$};
  \node[latent, above=of What] (SigmaW) {$\alpha_k^m$};
  \node[latent, above=of crossW, xshift=1.5cm] (SW) {$s_{k,d}^m$};
  \node[latent, above=of SW] (ThetaW) {$\theta_{k}^m$};

  % Connect the nodes
  \edge {Z,W, Tau} {Y}; %
  \edge {ThetaZ} {SZ};
  \edge {SigmaZ} {Zhat};
  \edge {ThetaW} {SW};
  \edge {SigmaW} {What};
  \edge[opacity=0.4] {Tau2} {Y}

  \factoredge {SZ, Zhat} {crossZ} {Z};
  \factoredge {SW, What} {crossW} {W};


  % Plates
  \plate[] {plateK} {(Z)(W)(SZ)(Zhat)(SW)(What)(ThetaZ)(ThetaW)} {$K$};
  \plate[color=colN, fill=colN, fill opacity=0.1] {plateN} {(Y)(Z)(crossZ)(Zhat)(SZ)(plateK.south west)} {\color{colN} $N$};
  \plate[color=colD ,fill=colD, fill opacity=0.1] {plateD} {(Y)(W)(Tau)(crossW)(What)(SW)(plateK.south east) (plateN.south east)} {\color{colD}$D_m$};
  \plate[color=colM,fill=colM, fill opacity=0.1] {plateM} {(plateD)(ThetaW)(plateK.north east)} {\color{colM}$M$};
\end{tikzpicture}

  \end{center}
  \caption{Full biofam model minus the cluster nodes}
\end{figure}

\begin{figure}
  \begin{center}
  	  % model-binomial-probit.tex
%

% Augmented Binomial Probit regression model

%\beginpgfgraphicnamed{model-binomial-probit}

\definecolor{colD}{rgb}{0.2, 0.2, 0.6}
\definecolor{colM}{rgb}{0.0, 0.5, 0.0}
\definecolor{colN}{rgb}{0.5, 0.0, 0.13}
\definecolor{colG}{rgb}{1.0, 0.65, 0.0}
\newcommand\op{0.25}
\colorlet{shadecolor}{black!25}


\begin{tikzpicture}
  % Define nodes:
  % matrix factorisation level
  \node[obs]   (Y) {$y_{n,d}^{m,g}$};
  \node[latent, above=of Y, xshift=-1.5cm] (Z) {$z^g_{n,k}$};
  \node[latent, above=of Y, xshift=1.5cm] (W) {$w_{k,d}^m$};
  \node[latent, xshift=1.5cm] (Tau) {$\tau_{d}^m$};

  % \node[opacity=0.4,latent, xshift=-1.5cm] (Tau2) {$\tau_{n}$};

  % parents of Z
  \node[det, above=of Z] (crossZ) {$\times$};
  \node[latent, above=of crossZ] (Zhat) {$\hat{z}^g_{n,k}$};
  \node[latent, above=of Zhat] (SigmaZ) {$\alpha^g_k$};
  \node[latent, above=of crossZ, xshift=-1.5cm] (SZ) {$s^g_{n,k}$};
  \node[latent, above=of SZ] (ThetaZ) {$\theta^g_{k}$};

  % parents of W
  \node[det, above=of W] (crossW) {$\times$};
  \node[latent, above=of crossW] (What) {$\hat{w}_{k,d}^m$};
  \node[latent, above=of What] (SigmaW) {$\alpha_k^m$};
  \node[latent, above=of crossW, xshift=1.5cm] (SW) {$s_{k,d}^m$};
  \node[latent, above=of SW] (ThetaW) {$\theta_{k}^m$};

  % Connect the nodes
  \edge {Z,W, Tau} {Y}; %
  \edge {ThetaZ} {SZ};
  \edge {SigmaZ} {Zhat};
  \edge {ThetaW} {SW};
  \edge {SigmaW} {What};
  % \edge[opacity=0.4] {Tau2} {Y}

  \factoredge {SZ, Zhat} {crossZ} {Z};
  \factoredge {SW, What} {crossW} {W};


  % Plates
  %factors
  \plate[] {plateK} {(Z)(W)(SZ)(Zhat)(SW)(What)(ThetaZ)(ThetaW)} {$K$};

  % sample side
  \plate[color=colN, fill=colN, fill opacity=0.1] {plateN} {(Y)(Z)(crossZ)(Zhat)(SZ)(plateK.south west)} {\color{colN} $N_g$};
  \plate[color=colG,fill=colG, fill opacity=0.1] {plateG} {(plateN)(plateK.north west)} {\color{colG}$G$};

  % feature side
  \plate[color=colD ,fill=colD, fill opacity=0.1] {plateD} {(Y)(W)(Tau)(crossW)(What)(SW)(plateK.south east) (plateG.south east)} {\color{colD}$D_m$};
  \plate[color=colM,fill=colM, fill opacity=0.1] {plateM} {(plateD)(ThetaW)(plateK.north east)} {\color{colM}$M$};
\end{tikzpicture}

  \end{center}
  \caption{New full biofam model: bi-group factor analysis}
\end{figure}


% \begin{center}
% 	  % model-binomial-probit.tex
%

% Augmented Binomial Probit regression model

%\beginpgfgraphicnamed{model-binomial-probit}

\definecolor{colD}{rgb}{0.2, 0.2, 0.6}
\definecolor{colM}{rgb}{0.0, 0.5, 0.0}
\definecolor{colN}{rgb}{0.5, 0.0, 0.13}
\definecolor{colG}{rgb}{1.0, 0.65, 0.0}
\newcommand\op{0.25}
\colorlet{shadecolor}{black!25}



\begin{tikzpicture}
  % Define nodes:
  % matrix factorisation level
  \node[obs]   (Y) {$y_{n,d}^m$};
  \node[latent, above=of Y, xshift=-1.5cm, yshift=0.5cm] (Z) {$z_{n,k}$};
  \node[latent, above=of Y, xshift=1.5cm] (W) {$w_{k,d}^m$};
  \node[latent, xshift=1.5cm] (Tau) {$\tau_{d}^m$};

  \node[opacity=0.4,latent, xshift=-1.5cm] (Tau2) {$\tau_{n}$};

  % parents of Z
  \node[det, above=of Z] (crossZ) {$\times$};
  \node[latent, above=of crossZ] (Zhat) {$\hat{z}_{n,k}^{\ }$};
  \node[latent, above=of Zhat] (SigmaZ) {$\Sigma_k$};
  \node[latent, above=of crossZ, xshift=-1.5cm] (SZ) {$s_{n,k}$};
  \node[latent, above=of SZ] (ThetaZ) {$\theta_{k}$};
  \node[latent, above=of Zhat, xshift=1.5cm, yshift=0.8cm] (muZ) {$\mu_{k, c}$};

  % parents of W
  \node[det, above=of W] (crossW) {$\times$};
  \node[latent, above=of crossW] (What) {$\hat{w}_{k,d}^m$};
  \node[latent, above=of What] (SigmaW) {$\Sigma_k^m$};
  \node[latent, above=of crossW, xshift=1.5cm] (SW) {$s_{k,d}^m$};
  \node[latent, above=of SW] (ThetaW) {$\theta_{k}^m$};
  \node[latent, above=of What, xshift=-1.5cm] (muW) {$\mu^m_{k, c}$};

  % Connect the nodes
  \edge {Z,W, Tau} {Y}; %
  \edge {ThetaZ} {SZ};
  \edge {SigmaZ} {Zhat};
  \edge {ThetaW} {SW};
  \edge {SigmaW} {What};
  \edge {muZ} {Zhat};
  \edge {muW} {What};
  \edge[opacity=0.4] {Tau2} {Y}

  \factoredge {SZ, Zhat} {crossZ} {Z};
  \factoredge {SW, What} {crossW} {W};


  % Plates
  % cluster plate
  \plate {plateC} {(muZ)(muW)} {$C$};
  \plate[color=colN, fill=colN, fill opacity=0.1] {plateN} {(Y)(Z)(crossZ)(Zhat)(SZ)} {\color{colN} $N$};
  \plate[color=colD ,fill=colD, fill opacity=0.1] {plateD} {(Y)(W)(Tau)(crossW)(What)(SW)(plateN.south east)} {\color{colD}$D_m$};
  \plate[color=colM,fill=colM, fill opacity=0.1] {plateM} {(plateD)(ThetaW) (muW)} {\color{colM}$M$};
  \plate[] {plateK} {(Z)(W)(SZ)(Zhat)(SW)(What)(ThetaZ)(ThetaW)(muW) (plateN.north west) (plateM.north east) (plateD.north east) (plateC)} {$K$};

\end{tikzpicture}

% \end{center}

\begin{figure}
  \begin{center}
  	  \input{biofam_mofa}
  \end{center}
  \caption{MOFA}
\end{figure}

\begin{figure}
  \begin{center}
  	  \input{biofam_tmofa}
  \end{center}
  \caption{transposed mofa for the combined analysis of multiple batches / tissues / cell types ... We keep the group wise ARD prior to help distinguish shared factors to factors which are unique to specific tissues}
\end{figure}


\begin{figure}
  \begin{center}
    % model-binomial-probit.tex
%

% Augmented Binomial Probit regression model

%\beginpgfgraphicnamed{model-binomial-probit}

\definecolor{colD}{rgb}{0.2, 0.2, 0.6}
\definecolor{colM}{rgb}{0.0, 0.5, 0.0}
\definecolor{colN}{rgb}{0.5, 0.0, 0.13}
\definecolor{colG}{rgb}{1.0, 0.65, 0.0}
\newcommand\op{0.25}
\colorlet{shadecolor}{black!25}


% In order to move back to a model with no cluster plate, remove the relevant commands, the relevant plate and correct the xshift for Z

\begin{tikzpicture}
  % Define nodes:
  % matrix factorisation level
  \node[obs]   (Y) {$y_{n,d}^m$};
  \node[latent, above=of Y, xshift=-1.5cm] (Z) {$z_{n,k}$};
  \node[latent, above=of Y, xshift=1.5cm] (W) {$w_{k,d}^m$};
  \node[latent, xshift=1.5cm] (Tau) {$\tau_{d}^m$};

  \node[opacity=\op,latent, xshift=-1.5cm] (Tau2) {$\tau_{n}$};

  % parents of Z
  \node[det, above=of Z] (crossZ) {$\times$};
  \node[latent, above=of crossZ] (Zhat) {$\hat{z}_{n,k}^{\ }$};
  \node[latent, above=of Zhat] (SigmaZ) {$\alpha_k$};
  \node[latent, above=of crossZ, xshift=-1.5cm] (SZ) {$s_{n,k}$};
  \node[latent, above=of SZ] (ThetaZ) {$\theta_{k}$};

  % parents of W
  \node[opacity=\op, det, above=of W] (crossW) {$\times$};
  \node[opacity=\op, latent, above=of crossW] (What) {$\hat{w}_{k,d}^m$};
  \node[opacity=\op, latent, above=of What] (SigmaW) {$\alpha_k^m$};
  \node[opacity=\op, latent, above=of crossW, xshift=1.5cm] (SW) {$s_{k,d}^m$};
  \node[opacity=\op, latent, above=of SW] (ThetaW) {$\theta_{k}^m$};

  % Connect the nodes
  \edge {Z,W, Tau} {Y}; %
  \edge {ThetaZ} {SZ};
  \edge {SigmaZ} {Zhat};
  \edge[opacity=\op] {ThetaW} {SW};
  \edge[opacity=\op] {SigmaW} {What};
  \edge[opacity=\op] {Tau2} {Y}

  \factoredge {SZ, Zhat} {crossZ} {Z};
  \factoredge[opacity=\op] {SW} {crossW} {W};
  \factoredge[opacity=\op] {What} {crossW} {W};

  % cluster plate
  % \node[opacity=\op, latent, above=of What, xshift=-1.3cm] (muW) {$\mu^m_{k, c}$};
  % \node[opacity=\op, latent, above=of Zhat, xshift=1.3cm] (muZ) {$\mu_{k, c}$};
  % \edge[opacity=\op] {muZ} {Zhat};
  % \edge[opacity=\op] {muW} {What};
  % \plate[opacity=\op] {plateC} {(muZ)(muW)} {$C$};

  % Plates
  % cluster plate
  \plate[] {plateK} {(Z)(W)(SZ)(Zhat)(SW)(What)(ThetaZ)(ThetaW)} {$K$};
  \plate[color=colN, fill=colN, fill opacity=0.1] {plateN} {(Y)(Z)(crossZ)(Zhat)(SZ)(plateK.south west)} {\color{colN} $N$};
  \plate[color=colD ,fill=colD, fill opacity=0.1] {plateD} {(Y)(W)(Tau)(crossW)(What)(SW)(plateK.south east) (plateN.south east)} {\color{colD}$D_m$};
  \plate[color=colM,fill=colM, fill opacity=0.1] {plateM} {(plateD)(ThetaW)(plateK.north east)} {\color{colM}$M$};
\end{tikzpicture}

  \end{center}
  \caption {Is this group ICA ? }
\end{figure}

\begin{figure}
  \begin{center}
    % model-binomial-probit.tex
%

% Augmented Binomial Probit regression model

%\beginpgfgraphicnamed{model-binomial-probit}

\definecolor{colD}{rgb}{0.2, 0.2, 0.6}
\definecolor{colM}{rgb}{0.0, 0.5, 0.0}
\definecolor{colN}{rgb}{0.5, 0.0, 0.13}
\definecolor{colG}{rgb}{1.0, 0.65, 0.0}
\newcommand\op{0.25}
\colorlet{shadecolor}{black!25}


% In order to move back to a model with no cluster plate, remove the relevant commands, the relevant plate and correct the xshift for Z

\begin{tikzpicture}
  % Define nodes:
  % matrix factorisation level
  \node[obs]   (Y) {$y_{n,d}^m$};
  \node[latent, above=of Y, xshift=-1.5cm] (Z) {$z_{n,k}$};
  \node[latent, above=of Y, xshift=1.5cm] (W) {$w_{k,d}^m$};
  \node[latent, xshift=1.5cm] (Tau) {$\tau_{d}^m$};

  \node[opacity=\op,latent, xshift=-1.5cm] (Tau2) {$\tau_{n}$};

  % parents of Z
  \node[det, above=of Z] (crossZ) {$\times$};
  \node[latent, above=of crossZ] (Zhat) {$\hat{z}_{n,k}^{\ }$};
  \node[latent, above=of Zhat] (SigmaZ) {$\Sigma_k$};
  \node[opacity=\op, latent, above=of crossZ, xshift=-1.5cm] (SZ) {$s_{n,k}$};
  \node[opacity=\op, latent, above=of SZ] (ThetaZ) {$\theta_{k}$};

  % parents of W
  \node[det, above=of W] (crossW) {$\times$};
  \node[latent, above=of crossW] (What) {$\hat{w}_{k,d}^m$};
  \node[latent, above=of What] (SigmaW) {$\alpha_k^m$};
  \node[latent, above=of crossW, xshift=1.5cm] (SW) {$s_{k,d}^m$};
  \node[latent, above=of SW] (ThetaW) {$\theta_{k}^m$};

  % Connect the nodes
  \edge {Z,W, Tau} {Y}; %
  \edge[opacity=\op] {ThetaZ} {SZ};
  \edge {SigmaZ} {Zhat};
  \edge {ThetaW} {SW};
  \edge {SigmaW} {What};
  \edge[opacity=\op] {Tau2} {Y}

  \factoredge[opacity=\op] {SZ} {crossZ} {Z};
  \factoredge[] {Zhat} {crossZ} {Z};
  \factoredge {SW, What} {crossW} {W};

  % cluster plate
  % \node[latent, above=of What, xshift=-1.3cm, opacity=0.15] (muW) {$\mu^m_{k, c}$};
  % \node[latent, above=of Zhat, xshift=1.3cm, opacity=0.15] (muZ) {$\mu_{k, c}$};
  % \edge[opacity=\op] {muZ} {Zhat};
  % \edge[opacity=\op] {muW} {What};
  % \plate[] {plateC} {(muZ)(muW)} {$C$};

  % Plates
  \plate[] {plateK} {(Z)(W)(SZ)(Zhat)(SW)(What)(ThetaZ)(ThetaW)} {$K$};
  \plate[color=colN, fill=colN, fill opacity=0.1] {plateN} {(Y)(Z)(crossZ)(Zhat)(SZ)(plateK.south west)} {\color{colN} $N$};
  \plate[color=colD ,fill=colD, fill opacity=0.1] {plateD} {(Y)(W)(Tau)(crossW)(What)(SW)(plateK.south east) (plateN.south east)} {\color{colD}$D_m$};
  \plate[color=colM,fill=colM, fill opacity=0.1] {plateM} {(plateD)(ThetaW)(plateK.north east)} {\color{colM}$M$};
\end{tikzpicture}

  \end{center}
  \caption{SpatialFA. Here we use a group factor analysis model and add a spatial covariance prior $\Sigma_k(l_k)$ to the factors. $\left(\Sigma_k\right)_{i,j} = \exp\left(-d_{i,j}^2/2l_k^2\right)$, where $d_{i,j}$ is the euclidean distance between cells $i$ and $j$ The hyperparamters $l_k, \forall k$ are optimised with variational EM.}
\end{figure}


\begin{figure}
  \begin{center}
    % model-binomial-probit.tex
%

% Augmented Binomial Probit regression model

%\beginpgfgraphicnamed{model-binomial-probit}

\definecolor{colD}{rgb}{0.2, 0.2, 0.6}
\definecolor{colM}{rgb}{0.0, 0.5, 0.0}
\definecolor{colN}{rgb}{0.5, 0.0, 0.13}
\definecolor{colG}{rgb}{1.0, 0.65, 0.0}
\newcommand\op{0.25}
\colorlet{shadecolor}{black!25}


\begin{tikzpicture}
  % Define nodes:
  % matrix factorisation level
  \node[obs]   (Y) {$y_{n,d}^m$};
  \node[latent, above=of Y, xshift=-1.5cm] (Z) {$z_{n,k}$};
  \node[latent, above=of Y, xshift=1.5cm] (W) {$w_{k,d}$};
  \node[latent, xshift=1.5cm] (Tau) {$\tau_{d}$};

  \node[opacity=\op,latent, xshift=-1.5cm] (Tau2) {$\tau_{n}$};

  % parents of Z
  \node[opacity=\op,det, above=of Z] (crossZ) {$\times$};
  \node[opacity=\op,latent, above=of crossZ] (Zhat) {$\hat{z}_{n,k}^{\ }$};
  \node[opacity=\op,latent, above=of Zhat] (SigmaZ) {$\alpha_k$};
  \node[opacity=\op,latent, above=of crossZ, xshift=-1.5cm] (SZ) {$s_{n,k}$};
  \node[opacity=\op,latent, above=of SZ] (ThetaZ) {$\theta_{k}$};

  % parents of W
  \node[opacity=\op,det, above=of W] (crossW) {$\times$};
  \node[opacity=\op,latent, above=of crossW] (What) {$\hat{w}_{k,d}^m$};
  \node[opacity=\op,latent, above=of What] (SigmaW) {$\alpha_k^m$};
  \node[opacity=\op,latent, above=of crossW, xshift=1.5cm] (SW) {$s_{k,d}^m$};
  \node[opacity=\op,latent, above=of SW] (ThetaW) {$\theta_{k}^m$};

  % Connect the nodes
  \edge {Z,W, Tau} {Y}; %
  \edge[opacity=\op] {ThetaZ} {SZ};
  \edge[opacity=\op] {SigmaZ} {Zhat};
  \edge[opacity=\op] {ThetaW} {SW};
  \edge[opacity=\op] {SigmaW} {What};
  \edge[opacity=\op] {Tau2} {Y}

  \factoredge[opacity=\op] {SZ, Zhat} {crossZ} {Z};
  \factoredge[opacity=\op] {SW, What} {crossW} {W};


  % Plates
  % \plate[] {plateK} {(Z)(W)(SZ)(Zhat)(SW)(What)(ThetaZ)(ThetaW)} {$K$};
  % \plate[color=colN, fill=colN, fill opacity=0.1] {plateN} {(Y)(Z)(crossZ)(Zhat)(SZ)(plateK.south west)} {\color{colN} $N$};
  % \plate[color=colD ,fill=colD, fill opacity=0.1] {plateD} {(Y)(W)(Tau)(crossW)(What)(SW)(plateK.south east) (plateN.south east)} {\color{colD}$D$};
  % \plate[color=colM,fill=colM, fill opacity=0.1] {plateM} {(plateD)(ThetaW)(plateK.north east)} {\color{colM}$M$};

  \plate[] {plateK} {(Z)(W)} {$K$};
  \plate[color=colN, fill=colN, fill opacity=0.1] {plateN} {(Y)(Z)(plateK.south west)(plateK.north west)} {\color{colN} $N$};
  \plate[color=colD ,fill=colD, fill opacity=0.1] {plateD} {(Y)(W)(Tau)(plateK.south east) (plateN.south east) (plateK.north east) (plateN.north east)} {\color{colD}$D$};
\end{tikzpicture}

  \end{center}
  \caption{Simple factor analysis model }
\end{figure}



\end{document}
