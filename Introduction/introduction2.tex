\chapter{Integrative an analysis of multi-omics data}

\section{Introduction}
Decades of statistical research have established a solid framework for the exploratory analysis of (single) omics data. However, the rapid development of single-cell technologies is introducing unprecedented challenges for the statistical community, and novel computational methods need to be developed (or adapted) for interrogating the data generated \cite{Stegle2015}. In particular, integrative analyses that simultaneously pool information across multiple data modalities (-omics) and across multiple studies promise to deliver a more comprehensive insights into the complex variation that underly cellular populations \cite{Stuart2019,Colome-Tatche2018}.\\

\subsection{Challenges in data integration}
The joint analysis of multiple data sources must tackle numerous challenges, some of which are:

\begin{itemize}
	\item \textbf{Heterogeneous data modalities}: measurements collected using different techniques generally exhibit heterogeneous statistical properties and have to be modelled under different statistical assumptions. For example, combining count (i.e. gene expression) and binary traits (i.e.somatic mutations) under the same statistical framework is not a trivial task. In the statistics community, this is commonly refered to as the multi-view learning problem \cite{Xu2013,Li2016}.

	\item \textbf{Overfitting}: as the number of molecular layers increase but the number of samples remain limited, modelling strategies need to account for the risk of overfitting, which can lead to poor generalisation performance. This is particularly important for complex (non-linear) methods that are receiving increasing interest in the machine learning community \cite{Zhang2018}.

	\item \textbf{Undesired sources of heterogeneity}: multi-omics data sets typically contain undesired sources of heterogeneity, both technical and biological \cite{Ritchie2015}. Prominent examples are batch effects or cell cycle variation, respectively. If not accounted for, such strong sources of variability can hidden the signal of interest \cite{Buettner2015}. Therefore, understanding and correcting for the underlying principal components of the data represent a critical step before other computational pipelines are applied \cite{Meng2016}.

	\item \textbf{Missing data}: a major problem in some single-cell methodologies is the large amounts of missing information. For example, in a typical single-cell bisulfite experiment less than 10\% of all CpG sites in the genome are measured \cite{Smallwood2014}. This imposes important challenges to some of the conventional statistical methods that do not handle missing information. Furthermore, different assays differ on how missing data is defined. For bisulfite sequencing methods, the missing values are clearly distinguishable from the observed values. However, for other methods such as scRNA-seq or scATAC-seq, the absence of sequence reads do not distinguish between the event that the genomic feature was not measured from that the readout was indeed zero \cite{Clark2018}.

	\item \textbf{Scalability}: as sequencing costs decrease and technologies improve, we anticipate that multi-modal data sets will follow a similar trend as scRNA-seq, where in the span of less than ten years the size of the experiments ranged from the order of tens to milions of cells \cite{Svensson2018}. The interrogation of exceptionally large data sets requires computational methods to scale accordingly.

	\item \textbf{Noise}: because of the small amounts of starting material, single-cell technologies are inherently noisy and result in large amounts of technical noise \cite{Stegle2015}. Hence, in most cases, inspection of individual genes or cells tends to be unreliable. To overcome this challenge, computational frameworks are required to leverage information on the similarities between cells and/or genes to delinate the signal from noise in order to obtain reliable estimates \cite{Vallejos2015}. Prominent examples are (empirical) Bayesian approaches that are able to borrow information across cells and/or genes and propagate uncertainity when doing inference and predictions \cite{Kharchenko2014}.

	\item \textbf{Data-driven hypothesis}: the exponential accumulation of biological data is empowering a new type of scientific method where the sheer data itself guides the researcher to formulate specific hypothesis. Importantly, rather than inverting the direction of the scientific method from data to hypothesis, the exploitation of the information hidden in big data sets has the potential to make this relationship bidirectional. Consequently, there is a growing interest in the use of unsupervised methods that, under some specific (hypothesis-driven) assumptions, are able to extract biological insights from the data.

	% \textbf{Continuous nature of the data}:

\end{itemize}

\subsection{Overview of methods for data integration}
Several approaches for multi-omics data integration have been pursuit over the last years in order to address the gap between the increasing availability of multi-omic data and the lack of computational methods. These approaches can be broadly classified into two classes: local analysis and global analysis \cite{Ritchie2015}.

Local analysis refers to the study of hierarchical associations between individual features from different molecular layers. Prominent examples are genome-wide association studies (GWAS) in combination with expression quantitative trait loci (eQTL), methylation QTLs or protein QTLs \cite{VanDerWijst2018,Chen2016,Pierce2018,Bonder2016}. While eminently useful for characterising genetic variants, such association studies are inherently local and have a limited capacity to discover global maps of molecular heterogeneity that typically result from complex interactions between features. In addition, such approaches are challening in the multi-omics setting due to the massive multiple testing problem \cite{Sul2015}.

Global analysis on the other hand try to extract patterns from the full data set. This can be done by direct concatenation of all data modalities followed by the use of traditional statistical methods. Nonetheless, as outlined above, this approach is generally not suitable to accomodate different data types (i.e. binary, count, continuous and binomial data). In addition, this approach further inflates the dimensionality of the feature space, increasing the risk of overfitting and lessen interpretability.\\
Alternatives have been proposed that perform transformations on each data type before merging them into a common similarity network, e.g. using kernel or graph-based approaches \cite{Lanckriet2004, Wang2014}. However, in such approaches interpretability is lost on the feature level. Moreover, without an underlying generative model, they do not naturally handle incomplete data and are uncapable of doing predictions.

Arguably, some of the most succcessful global approaches are based on dimensionality reduction. Conventional methods include generalised versions of Canonical Correlation Analysis, Multiple coinertia analysis\cite{Hanafi2011} or Joint Non-negative Matrix Factorisation \cite{Liu2012}. The transition from single-view latent variable models to multi-view scenarios is described in more detail in \Cref{XX}. One of the most used latent variable models for multi-omics data integration is iCluster \cite{Shen2009, Mo2013}, a clustering method that will be discussed in more detail in \Cref{section:icluster}. However, as a non-generative clustering approach that involves expensive grid-search in its inference procedure, this model is limited by its inability to handle missing values ad limited scalability. Also, it results in artifical clusters in the presence of undesired sources of heterogeneity and when dealing with continuous data \cite{Argelaguet2018}.

Finally, while the previous methods are general statistical formulations that have been applied to a wide range of domains, some models have been designed to contain tailored \textit{a priori} assumptions given by the dogma of molecular biology \cite{Sass2013}. 


\subsection{Strategies for the integrative analysis of single-cell multi-modal data }
