\section{Introduction}

Recent technological advances have enabled single-cell technologies to explode in numbers, with some studies profiling more than a milion cells. Even in the more challening multi-modal domain, protocols are reaching the order of thousands of cells. The increasing availability of multi-modal data sets needs to be accompanied by the development of novel integrative strategies. Yet, given the high levels of missing informatio, the inherent amounts of technical noise and the potentially large number of cells, the integrative analysis of single-cell omics is one of the most challenging problems in computational biology. \\

In this PhD thesis we develop a new experimental and computational method for the integrative analysis of single-cell multi-omics data.\\

In Chapter 1 we discuss single-cell nucleosome, methylation and transcription sequencing (scNMT-seq), an experimental protocol for the genome-wide profiling of RNA expression, DNA methylation and chromatin accessibility in single cells. While some approaches have reported unbiased genome-wide measurements of up to two molecular layers, scNMT-seq allows, for the first time, the profiling of RNA expression coupled with multiple epigenetic layers at single cell resolution.\\
In a proof of concept application, we validate the quality of the readouts, followed by a comparison with similar technologies. Finally, we show how scNMT-seq can be used to study coordinated epigenetic and transcriptomic heterogeneity along a simple differentiation process.\\

In Chapter 2 we discuss Multi-Omics Factor Analysis (MOFA), a statistical framework for the integration of multi-omics data sets. MOFA is a latent variable model that offers principled approach to interrogate multi-omics data sets in a completely unsupervised manner, revealing the underlying sources of sample heterogeneity. Once the model is trained, the infered low-dimensional space can be queried using a toolkit of downstream analysis, including visualisation, clusteirng, imputation or prediction of clinical outcomes.\\
First, we validate the different features of the model using simulated data, including the scalability, sparsity assumptions and the non-gaussian likelihoods. Second, to demonstrate the potential of the method, we applied MOFA to a multi-omics study of 200 chronic lymphocytic leukaemia patients. In a quick unsupervised analysis, MOFA revealed the most important dimensions of disease heterogeneity, connected to clinical markers that are commonly used in practice. In a second application we show how MOFA can cope with noisy single-cell multi-modal data, identifying coordinated transcriptional and epigenetic changes along a differentiation process.

In Chapter 3 we apply scNMT-seq to study the role of epigenetic layers during mouse gastrulation, a critical embryonic stage that spans exit from pluripotency to primary germ layer specification. Gene expression dynamics during gastrulation have been characterised in detail, but the role of the epigenome in driving cell fate decisions remains poorly understood.\\
In this study we provide the first triple-omics roadmap of mouse gastrulation. Using MOFA, we perform an integrative study of all molecular layers, revealing novel insights on the dynamics of the epigenome. Notably, we show that cells commited to mesoderm and endoderm undergo widespread epigenetic rearrangements, driven by demethylation in enhancer marks and by concerted changes in chromatin accessibility. In contrast, the epigenetic landscape of ectoderm cells remains in a “default” state, resembling earlier stage epiblast cells is epigenetically established in the early epiblast. This work provides a comprehensive insight into the molecular logic for a hierarchical emergence of the primary germ layers, revealing underlying molecular constituents of the Waddington's landscape.\\

In Chapter 4 we propose an improved formulation of the MOFA framework presented in Chapter 2 with the aim of performing integrative analysis of large-scale (single-cell) data sets across multiple studies/conditions as well as data modalities.\\
To tailor MOFA to the statistical challenges of single-cell data, we introduce key methodological developments, including a fast stochastic variational inference framework, a new structured sparsity prior and the relaxation of the assumption of independent samples. All together, this allows MOFA to simultaneously disentangle heterogeneity across studes/conditions and data modalities in very large single-cell studies.\\
First, we benchmark the new features of the model using simulated data. Next, we use a single-cell DNA methylation data set of neurons from mouse frontal cortex to demonstrate how from a seemingly unimodal data set, one can investigate hypothesis using a multi-group and multi-view setting. Finally, we apply MOFA to the scNMT-seq data set generated in Chapter3, disentangling the sources of heterogeneity assocaited with early cell fate decisions.\\

Finally, Chapter 5 summarises this thesis and provides an outlook of future research.