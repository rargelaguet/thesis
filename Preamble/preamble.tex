% ****************************** Custom Margin *********************************

% Add `custommargin' in the document class options to use this section
% Set {innerside margin / outerside margin / topmargin / bottom margin}  and
% other page dimensions
\ifsetCustomMargin
  \RequirePackage[left=23mm,right=23mm,top=27mm,bottom=22mm]{geometry}
%  \RequirePackage[left=37mm,right=30mm,top=35mm,bottom=30mm]{geometry}
  \setFancyHdr % To apply fancy header after geometry package is loaded
\fi

% Define paragraph settings
\setlength{\parskip}{0.7em}  % space between paragraphs
\setlength{\parindent}{0pt}  % indentation

% Ragged bottom avoids extra whitespaces between paragraphs
\raggedbottom

% To remove the excess top spacing for enumeration, list and description
%\usepackage{enumitem}
%\setlist[enumerate,itemize,description]{topsep=0em}


% ************************* Algorithms and Pseudocode **************************
\usepackage{algorithm}
%\usepackage{algpseudocode}
%\usepackage[noend]{algpseudocode}


% ********************Captions and Hyperreferencing / URL **********************

% to support older versions of captions.sty
%\renewcommand{\figurename}{Fig.} 

% Captions for figures and subfigures
%\RequirePackage[small,bf]{caption} % This makes captions of figures use a boldfaced small font.
%\RequirePackage[labelsep=space,tableposition=top]{caption}
\RequirePackage[font={small}, margin=20pt, labelfont=bf]{caption}
\RequirePackage[font={footnotesize}, margin=20pt]{subcaption}


% *************************** Graphics and figures *****************************

%\usepackage{rotating}
%\usepackage{wrapfig}
\usepackage[section]{placeins}

% Uncomment the following two lines to force Latex to place the figure.
% Use [H] when including graphics. Note 'H' instead of 'h'
% \usepackage{float}
% \restylefloat{figure}

% Subcaption package is also available in the sty folder you can use that by
% uncommenting the following line
% This is for people stuck with older versions of texlive
%\usepackage{sty/caption/subcaption}
\usepackage{subcaption}

% ********************************** Tables ************************************
\usepackage{booktabs} % For professional looking tables
\usepackage{multirow}

\usepackage{multicol}
\usepackage{longtable}
\usepackage{tabularx}


% *********************************** SI Units *********************************

%\usepackage{siunitx}

% ******************************* Line Spacing *********************************

% Default is one-half line spacing as per the University guidelines

% \doublespacing
\onehalfspacing
% \singlespacing


% ************************ Formatting / Footnote *******************************

% Don't break enumeration (etc.) across pages in an ugly manner (default 10000)
%\clubpenalty=500
%\widowpenalty=500

%\usepackage[perpage]{footmisc} %Range of footnote options


% *************************** Graphical models ********************

\usepackage{tikz}
\usetikzlibrary{positioning}
\usetikzlibrary{bayesnet}

% *************************** Bibliography  and References ********************

% References
%\usepackage{hyperref}             % Hyperlink in references
\usepackage[nameinlink]{cleveref}  % Clever references (Cref)

% Per-chapter numbering of figures
\usepackage{chngcntr}
\counterwithin{figure}{chapter}


% Bibliography
% \RequirePackage[backend=biber, style=numeric-comp, citestyle=numeric, sorting=nty, natbib=true]{biblatex}
\RequirePackage[hyperref=true, url=false, backref=false, backend=biber, style=numeric,giveninits=true]{biblatex}

% Add bib resources
\addbibresource{references/references_chapter1.bib}
\addbibresource{references/references_chapter2.bib}
\addbibresource{references/references_chapter3.bib}
\addbibresource{references/references_chapter4.bib}

% changes the default name `Bibliography` -> `References'
% \renewcommand{\bibname}{References}

% ******************************************************************************
% ************************* User Defined Commands ******************************
% ******************************************************************************

% *********** To change the name of Table of Contents / LOF and LOT ************

%\renewcommand{\contentsname}{My Table of Contents}
%\renewcommand{\listfigurename}{My List of Figures}
%\renewcommand{\listtablename}{My List of Tables}

% ********************** TOC depth and numbering depth *************************

% Spacing in the TOC
\usepackage{tocloft}
\renewcommand\cftchapafterpnum{\vskip1pt}
\renewcommand\cftsecafterpnum{\vskip1pt}

% Depth in the TOC
% The sectioning levels have the following numbers:
% -1 part     1 section     3 subsubsection  5 subparagraph
%  0 chapter  2 subsection  4 paragraph

% The "tocdepth" value determines to which level the sectioning commands are printed in the ToC
\setcounter{tocdepth}{2}  % Default was 2

% The "secnumdepth" value determines up to what level the sectioning titles are numbered
\setcounter{secnumdepth}{2}  % Default was 2



% ******************************* Nomenclature *********************************

% To change the name of the Nomenclature section, uncomment the following line

%\renewcommand{\nomname}{Symbols}


% ********************************* Appendix ***********************************

% The default value of both \appendixtocname and \appendixpagename is `Appendices'. These names can all be changed via:

\renewcommand{\appendixtocname}{List of appendices}
\renewcommand{\appendixname}{Appendix}

% *********************** Configure Draft Mode **********************************

% Uncomment to disable figures in `draft'
%\setkeys{Gin}{draft=true}  % set draft to false to enable figures in `draft'

% These options are active only during the draft mode
% Default text is "Draft"
%\SetDraftText{DRAFT}

% Default Watermark location is top. Location (top/bottom)
%\SetDraftWMPosition{bottom}

% Draft Version - default is v1.0
%\SetDraftVersion{v1.1}

% Draft Text grayscale value (should be between 0-black and 1-white)
% Default value is 0.75
%\SetDraftGrayScale{0.8}


% ******************************** Todo Notes **********************************
%% Uncomment the following lines to have todonotes.

%\ifsetDraft
%	\usepackage[colorinlistoftodos]{todonotes}
%	\newcommand{\mynote}[1]{\todo[author=kks32,size=\small,inline,color=green!40]{#1}}
%\else
%	\newcommand{\mynote}[1]{}
%	\newcommand{\listoftodos}{}
%\fi

% Example todo: \mynote{Hey! I have a note}

% ******************* Better enumeration *************
\usepackage{enumitem}
